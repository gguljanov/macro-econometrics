\documentclass[11pt]{article}

% === Packages ===
\usepackage[a-1b]{pdfx}

\usepackage[utf8]{inputenc}

\usepackage{amssymb}
\usepackage{amsmath}
\usepackage{amsthm}
\newtheorem{definition}{Definition}[section]

\usepackage{amsfonts}

\usepackage{nicefrac}

\usepackage{graphicx}

\usepackage{multirow}

% For using \toprule and \midrule in the tables
\usepackage{booktabs}

\usepackage{caption}

\usepackage{dcolumn}

\usepackage{siunitx}

% Change the title format
\usepackage[explicit]{titlesec}
\titleformat{\section}{\normalfont\Large\bfseries}{}{0em}{%
    Ch. \thesection ~ -- ~ #1%
}
\titleformat{\subsection}{\normalfont\large\bfseries}{}{0em}{%
    Ex. \thesubsection ~ -- ~ #1%
}

% Enumerating items
\usepackage{enumitem}
\setlist[enumerate]{font=\bfseries, label=(\alph*)} %, itemsep=0.5cm}

% Set bookmarks for sections, equations etc.
% \usepackage[bookmarks, bookmarksopen, bookmarksdepth=3]{hyperref}
\hypersetup{bookmarksnumbered}

% % Different colors for font and the background
% \usepackage[dvipsnames]{xcolor}

% Cancel by striking through
\usepackage{cancel}

% Set the space between lines
\usepackage{setspace}

% Include code chunks beautifully
\usepackage{listingsutf8}
\lstset{
    language=R,
    frame=tb,
    basicstyle=\ttfamily\footnotesize,
    alsoletter={.},
    breaklines=true,
    numbers=left,
    numberstyle=\tiny\color{gray},
    showstringspaces=false
}

% References
\usepackage[citestyle=alphabetic, bibstyle=authortitle]{biblatex}
\addbibresource{biblio.bib}

% \today and \currentime commands
\usepackage{datetime}

% \usepackage{todonotes}
% \usepackage{tikz}

% Space between lines
\usepackage{setspace}
\linespread{1.2}

% Page margins
\usepackage[top=2.5cm, bottom=2.5cm, left=2.5cm, right=2cm]{geometry}

% Space between paragraphs
\usepackage{parskip}

% Paragraph indentation
\setlength{\parindent}{0pt}


% === Commands ===
% Expectation operator
\DeclareMathOperator*{\E}{\mathbb{E}}
\DeclareMathOperator*{\Var}{\mathbb{V}ar}
\DeclareMathOperator*{\Cov}{\mathbb{C}ov}

% Colored cancel
\newcommand*{\Ccancel}[2][black]{
    \renewcommand{\CancelColor}{\color{#1}}{\cancel{#2}}
}

\newtheorem*{definition*}{Definition}

% Writing ß with \3
\let\3 = \ss

\newcommand*{\iid}{\overset{\text{iid}}{\sim}}

% Writing code and output
\newcommand*{\writeCode}{
    \vspace{0.3cm}
    \textcolor{black}{\underline{\textbf{Code}}}:
}

\newcommand*{\writeOutput}{
    \vspace{0.3cm}
    \textcolor{black}{\underline{\textbf{Output}}}:
    \vspace{-0.8cm}
}


% === Others ===
% Allows box-overflow amount before a warning message is issued
\hfuzz=6000pt

% Allows to control the space added before the math equations
\abovedisplayskip=-0.2cm
\abovedisplayshortskip=-0.2cm

% Allows to control the space added after the math equations
\belowdisplayskip=-0.2cm
\belowdisplayshortskip=-0.2cm

% Allows the align math expression be broken into pages at the output pdf
\allowdisplaybreaks

% Check mark and X mark
\usepackage{pifont}

\newcommand{\cmark}{\ding{51}}
\newcommand{\xmark}{\ding{55}}

% Wide hat
\usepackage{scalerel, stackengine}
\stackMath

\newcommand\reallywidehat[1]{%
    \savestack{\tmpbox}{\stretchto{%
            \scaleto{%
                \scalerel*[\widthof{\ensuremath{#1}}]{%
                    \kern-.6pt\bigwedge\kern-.6pt%
                }{
                    \rule[-\textheight/2]{1ex}{\textheight}
                }%WIDTH-LIMITED BIG WEDGE
            }{\textheight}% 
        }{1ex}}%
    \stackon[1pt]{#1}{\tmpbox}%
}

%  === Abbreviations and Acronyms ===
\newcommand{\ols}{Ordinary Least Squares}
\newcommand{\ml}{Maximum Likelihood}

\newcommand*{\ar}{AR}
\newcommand*{\arp}[1][p]{AR(#1)}

\newcommand*{\var}{VAR}
\newcommand*{\varp}[1][p]{VAR(#1)}

\newcommand*{\svar}{SVAR}
\newcommand*{\svarp}[1][p]{SVAR(#1)}

\newcommand*{\ma}{MA}
\newcommand*{\maq}[1][q]{MA(#1)}

\newcommand*{\vma}{VMA}
\newcommand*{\vmaq}[1][q]{VMA(#1)}

\newcommand*{\arma}{ARMA}
\newcommand*{\armapq}[1][p, q]{ARMA(#1)}

\newcommand*{\varma}{VARMA}
\newcommand*{\varmapq}[1][p, q]{VARMA(#1)}

\newcommand*{\arima}{ARIMA}
\newcommand*{\arimapdq}[1][p, d, q]{ARIMA(#1)}

\newcommand*{\arch}{ARCH}
\newcommand*{\archp}[1][p]{ARCH(#1)}

\newcommand*{\garch}{GARCH}
\newcommand*{\garchpq}[1][p, q]{GARCH(#1)}


% === Solutions open/closed ===
% \usepackage{answers}
\usepackage[nosolutionfiles]{answers}

\renewcommand{\solutionstyle}[1]{\textbf{Sol. #1}:\hspace*{\textwidth}}
% \newcommand\preSolution{\hfill}
\newcommand\postSolution{\hfill\textbf{End of solution}\vspace*{0.8cm}}

\Newassociation{sol}{Solution}{ans}

% === Font ===
\usepackage{newpxtext, newpxmath}


% === Document begins ===
\begin{document}

\Opensolutionfile{ans}[ans]

\title{Macro-Econometrics --- Exercise Book}

\author{}

\date{This version: \today ~\currenttime}

\maketitle

% === References ===
The references for this tutorial are
\begin{enumerate}[label=\roman*.]
    \item \cite{MartinHurnHarris-2012}
    \item \cite{Mutschler-2018-github_repo}
    \item Other explicit citations within the text
\end{enumerate}

% % === Exercises ===
% % Intro
% \subsection{Introduction to Macroeconometrics\texorpdfstring{%
%         \protect\footnote{Taken from \cite[][see the section no 1, titled ``Macroeconometrics''.]{Mutschler-2018-github_repo}}%
%     }{}}
% Broadly define the term and research topics of \emph{Macroeconometrics}.

\begin{sol}
    Definition of term:
    \begin{enumerate}[label=]
        \item Combination of modern theoretical macroeconomics (the study of aggregated variables such as economic growth, unemployment and inflation by means of structural macroeconomic models) and econometric methods (the application of formal statistical methods in empirical economics).
    \end{enumerate}

    Research topics:
    \begin{enumerate}[label=\roman*.]
        \item How to identify sources of fluctuations, e.g. how important are monetary policy shocks as opposed to other shocks for movements in aggregate output? [forecast error variance decomposition]

        \item Understand propagation of shocks, e.g. what happens to aggregate hours worked over the next two years in response to a technology shock in the current quarter? [impulse response function]

        \item Forecasting the future, e.g. how will inflation and output growth rates evolve over next eight quarters. [forecasting]

        \item Predict effect of policy changes, e.g. how will output and inflation respond to an unanticipated change in nominal interest rate? [impulse response function and forecast scenarios]

        \item Structural changes in the economy, e.g. has monetary policy changed in the early 1980s, why did volatility of many macroeconomic time series drop in the mid 1980s, [historical decomposition]

        \item How much of the recession of 1982  would have deepened had monetary policymakers not responded to output growth at all. [policy counterfactual]
    \end{enumerate}
\end{sol}

% % From univariate to multivariate models
% \subsection{From univariate to multivariate models}
% \begin{enumerate}
    \item Consider the following two \armapq[1, 1]{} models, for inflation and output, respectively:
          \[ \pi_t = 0.3 + 0.8 \pi_{t-1} + \epsilon_t - 0.1 \epsilon_{t-1} \]
          and
          \[ o_t = -0.64 o_{t-1} + \eta_t + 0.2 \eta_{t-1}. \]

          Write these two \armapq[1, 1]{} models as \varmapq[1, 1]{}.

          \begin{sol}
              \begin{align*}
                  \underbrace{
                      \begin{pmatrix}
                          \pi_t \\ o_t
                      \end{pmatrix}
                  }_{
                      y_t
                  }
                  = \underbrace{
                      \begin{pmatrix} 0.3 \\ 0 \end{pmatrix}
                  }_{
                      \mu
                  }
                  + \underbrace{
                      \begin{bmatrix}
                          0.8 & 0    \\
                          0   & -0.6
                      \end{bmatrix}
                  }_{
                      \Phi_1
                  }
                  \underbrace{
                      \begin{pmatrix}
                          \pi_{t-1} \\ o_{t-1}
                      \end{pmatrix}
                  }_{
                  y_{t-1}
                  }
                  + \underbrace{
                      \begin{pmatrix}
                          \epsilon_{t} \\ \eta_{t}
                      \end{pmatrix}
                  }_{
                      \nu_t
                  }
                  + \underbrace{
                      \begin{bmatrix}
                          -0.1 & 0   \\
                          0    & 0.2
                      \end{bmatrix}
                  }_{
                      \Psi_1
                  }
                  \underbrace{
                      \begin{pmatrix}
                          \epsilon_{t-1} \\ \eta_{t-1}
                      \end{pmatrix}
                  }_{
                  \nu_{t-1}
                  }
              \end{align*}
          \end{sol}

    \item Consider a \varp[1]{} model with the following parameter matrix
          \begin{align*}
              \mu    & = \begin{pmatrix} 0 \\ -0.04 \end{pmatrix}                    \\
              \Phi_1 & = \begin{bmatrix} -0.81 & 0.26 \\ -0.34 & -1.53 \end{bmatrix}
          \end{align*}

          Write the above \varp[1]{} model as two univariate models. Are the resulting univariate models \arp[1]{}?

          \begin{sol}
              Let us first write the \varmapq[1, 1]{} model more clearly:
              \begin{align*}
                  \begin{pmatrix} x_{t} \\ z_{t} \end{pmatrix}
                  = \begin{pmatrix} 0 \\ -0.04 \end{pmatrix}
                  + \begin{bmatrix} -0.81 & 0.26 \\ -0.34 & -1.53 \end{bmatrix}
                  \begin{pmatrix} x_{t-1} \\ z_{t-1} \end{pmatrix}
                  + \begin{pmatrix} \epsilon_{t} \\ \eta_{t} \end{pmatrix}
              \end{align*}

              The two univariate models would then be as follows:
              \begin{align*}
                  x_t & = -0.81 x_{t-1} + 0.26 z_{t-1} + \epsilon_t    \\
                  z_t & = -0.04 - 0.34 x_{t-1} - 1.53 z_{t-1} + \eta_t
              \end{align*}

              The univariate models above are not exactly \arp[1]{} model, but are rather, Autoregressive Distributed Lag models, because they contain not only their own lags but also lags of the other variable.
          \end{sol}

    \item Why are we concerned with multivariate time series? For example, why do we model the VAR(1) process
          \begin{align*}
              \begin{pmatrix} y_{1,t} \\ y_{2,t} \end{pmatrix}
              = \begin{pmatrix} \mu_1 \\ \mu_2 \end{pmatrix}
              + \begin{pmatrix}
                    \phi_{11} & \phi_{12} \\
                    \phi_{21} & \phi_{22}
                \end{pmatrix}
              \begin{pmatrix} y_{1, {t-1}} \\ y_{2, {t-1}} \end{pmatrix}
              + \begin{pmatrix} \nu_{1, t} \\ \nu_{2, t} \end{pmatrix}
          \end{align*}
          simultaneously, instead of two models for each variable separately?\footnote{Taken from \cite[][see the section 16, titled ``Understanding multivariate time series concepts"]{Mutschler-2018-github_repo}}

          \begin{sol}
              For the specification of multi-equation models, we require a clear distinction between exogenous and endogenous variables. In economic theory, this is often not clear or arbitrarily made in practice. Vectorautoregressive models do not need this distinction, they can rather be understood as a dynamic version of a simultaneous multi-equation model. This corresponds to reality, because economic variables are generated by dynamic processes, and often are interdependent.

              Therefore, VAR models provide a powerful instrument. They also take into account things like non-stationarity (cointegration and long-term equilibria), as well as the analysis of dynamics of random shocks / impulses.

              An Example: we are also able to consider correlations between $\nu_{1, t}$ and $\nu_{2, t}$.

              Lastly, VAR models tend to have better predictive power than multi-equation models and are often used as a benchmark for different forecasting models.
          \end{sol}

    \item Consider the following \svarp[1]{} model:
          \begin{align*}
              \underbrace{
                  \begin{bmatrix}
                      b_{0, 1, 1} & b_{0, 1, 2} \\
                      b_{0, 2, 1} & b_{0, 2, 2}
                  \end{bmatrix}
              }_{
                  B_0
              } y_t
              = \underbrace{
                  \begin{bmatrix}
                      b_{1, 1, 1} & b_{1, 1, 2} \\
                      b_{1, 2, 1} & b_{1, 2, 2}
                  \end{bmatrix}
              }_{B_1} y_{t-1}
              + u_t
          \end{align*}

          Write the above model as \varp[1]{}.

          \begin{sol}
              Let us first invert the $B_0$ matrix:
              \begin{align*}
                  B_0^{-1} & = \underbrace{
                      \frac{1}{
                          b_{0, 1, 1} b_{0, 2, 2} - b_{0, 1, 2} b_{0, 2, 1}
                      }
                  }_{:= c}
                  \begin{bmatrix}
                      b_{0, 2, 2}  & -b_{0, 1, 2} \\
                      -b_{0, 2, 1} & b_{0, 1, 1}
                  \end{bmatrix} \\
                           & =
                  \begin{bmatrix}
                      c b_{0, 2, 2}  & -c b_{0, 1, 2} \\
                      -c b_{0, 2, 1} & c b_{0, 1, 1}
                  \end{bmatrix}
              \end{align*}

              \begin{align*}
                  B_0 y_t & = B_1 y_{t-1} + u_t
                  \\[0.4cm]
                  y_t     & = B_0^{-1} B_1 y_{t-1} + B_0^{-1} u_t
                  \\[0.4cm]
                          & = \begin{bmatrix} % B_0^{-1}
                                  c b_{0, 2, 2}  & -c b_{0, 1, 2} \\
                                  -c b_{0, 2, 1} & c b_{0, 1, 1}
                              \end{bmatrix}
                  \begin{bmatrix} % B_1
                      b_{1, 1, 1} & b_{1, 1, 2} \\
                      b_{1, 2, 1} & b_{1, 2, 2}
                  \end{bmatrix}
                  y_{t-1}
                  +
                  \begin{bmatrix} % B_0^{-1}
                      c b_{0, 2, 2}  & -c b_{0, 1, 2} \\
                      -c b_{0, 2, 1} & c b_{0, 1, 1}
                  \end{bmatrix}
                  \begin{pmatrix}
                      u_{1, t} \\
                      u_{2, t}
                  \end{pmatrix}
                  \\[0.4cm]
                          & =
                  \underbrace{
                      \begin{bmatrix} % Phi_1
                          c b_{0, 2, 2} b_{1, 1, 1}
                          - c b_{0, 1, 2} b_{1, 2, 1}
                           & c b_{0, 2, 2} b_{1, 1, 2}
                          - c b_{0, 1, 2} b_{1, 2, 2}
                          \\
                          -c b_{0, 2, 1} b_{1, 1, 1}
                          + c b_{0, 1, 1} b_{1, 2, 1}
                           & -c b_{0, 2, 1} b_{1, 1, 2}
                          + c b_{0, 1, 1} b_{1, 2, 2}
                      \end{bmatrix}
                  }_{\Phi_1}
                  y_{t-1}
                  +
                  \underbrace{
                      \begin{bmatrix} % nu_t
                          c b_{0, 2, 2} u_{1, t} - c b_{0, 1, 2} u_{2, t} \\
                          -c b_{0, 2, 1} u_{1, 6} + c b_{0, 1, 1} u_{2, t}
                      \end{bmatrix}
                  }_{\nu_t}
                  \\[0.4cm]
                          & = \Phi_1 y_{t-1} + \nu_t
              \end{align*}
          \end{sol}

    \item Using R, simulate and plot $250$ observations from the following \varp[1]{} model:
          \[ y_t = \Phi_1 y_{t-1} + \nu_t \]
          where,
          \begin{align*}
              y_0    & = \begin{pmatrix}
                             0 \\
                             0
                         \end{pmatrix}    \\[0.5cm]
              \Phi_1 & = \begin{bmatrix}
                             -0.66 & 0.26  \\
                             -0.30 & -0.58
                         \end{bmatrix}    \\[0.5cm]
              \nu_t  & \sim N(0, I_{2, 2})
          \end{align*}

          \begin{sol}
              \lstinputlisting{../R-files/simulate_var1_250.R}
          \end{sol}

    \item Using R, simulate and plot $250$ observations from the following \svarp[1]{} models:
          \[ B_0 y_t = B_1 y_{t-1} + u_t \]
          where,
          \begin{align*}
              y_0 & = \begin{pmatrix}
                          0 \\
                          0
                      \end{pmatrix}    \\[0.5cm]
              B_0 & = \begin{bmatrix}
                          -0.98 & -0.42 \\
                          0.49  & 1.11
                      \end{bmatrix}    \\[0.5cm]
              B_1 & = \begin{bmatrix}
                          -0.66 & 0.26  \\
                          -0.30 & -0.58
                      \end{bmatrix}    \\[0.5cm]
              u_t & \sim N(0, I_{2, 2})
          \end{align*}

          \begin{sol}
              \lstinputlisting{../R-files/simulate_svar1_250.R}
          \end{sol}
\end{enumerate}


% % Dimensions in VAR models
% \subsection{Dimensions in \texorpdfstring{\varp{}}{VAR} models\texorpdfstring{%
%         \protect\footnote{Based on \cite[][see the section 17, titled ``Dimensions and VAR(1) representation'']{Mutschler-2018-github_repo}.}%
%     }{}
% }
% Let $y_t$ be a $N$-dimensional covariance stationary random vector. Consider the \varp[p]{}-process
\begin{align*}
    y_t = \mu + \sum_{i=1}^p \Phi_i y_{t-i} + \nu_t, \quad \nu \sim N(0, V)
\end{align*}

\begin{enumerate}
    \item What are the dimensions of $\mu$, $\Phi_i$ and $\nu_t$?

          \begin{sol}
              $\mu$ and $u_t$ are both $N$-dimensional vectors: $\mu, \nu_t \in \mathbb{R}^{N \times 1}$.

              $\Phi_i$ is a $N \times N$-Matrix.
          \end{sol}

    \item Consider a \varp[2]{} model with $N=4$ variables and a constant term. How many parameters do we need to estimate?

          \begin{sol}
              In general, there are $N + N^2 \times p + N (N + 1) / 2$:
              \begin{enumerate}[label=-]
                  \item $\mu$ contains $N$ parameters

                  \item $\Phi_i$ contains $N^2$ parameters and there are $p$ of them $\Rightarrow$ $p \times N^2$

                  \item The covariance matrix $V$ contains $N (N + 1) / 2$ parameters (Note: since the covariance matrix is symmetric, we should consider either only upper half or only lower half of the matrix, together with the diagonal)
              \end{enumerate}

              In our specific case the number of parameters is then
              \[ 4 + 4^2 \cdot 2 + 4 \cdot (4 + 1) / 2 = 46. \]
              That's a lot! Therefore we will try to restrict some parameters (e.g. set equal to zero) or consider only small VAR systems, e.g. $N=3$ and $p=1$.
          \end{sol}
\end{enumerate}


% % Dimensions in SVAR models
% \subsection{Dimensions in \texorpdfstring{\svarp{}}{SVAR} models\texorpdfstring{%
%         \protect\footnote{Based on \cite[][see the section 17, titled ``Dimensions and VAR(1) representation'']{Mutschler-2018-github_repo}.}%
%     }{}}
% Let $y_t$ be a $N$-dimensional covariance stationary random vector and $K$ be the number of relationships in $y_t$. Consider the \varp[p]{}-process
\begin{align*}
    B_0 y_t = \gamma + \sum_{i=1}^p B_i y_{t-i} + u_t, \quad u_t \sim N(0, D)
\end{align*}

\begin{enumerate}
    \item What are the dimensions of $\gamma$, $B_0$, $B_i$, and $u_t$?

          \begin{sol}
              $\gamma$ and $u_t$ are both $N$-dimensional vectors: $\mu, \nu_t \in \mathbb{R}^{N \times 1}$.

              $B_0$ and $B_i$ are $N \times N$-Matrices.
          \end{sol}

    \item Consider a \svarp[2]{} model with $N=4$ variables, $K=2$ relationships in $y_t$ and a constant term. How many parameters do we need to estimate?

          \begin{sol}
              In general, there are $NK + N + N^2 \times p + N$:
              \begin{enumerate}[label=-]
                  \item $B_0$ contains $N \times K$ parameters

                  \item $\gamma$ contains $N$ parameters

                  \item $B_i$ contains $N^2$ parameters and there are $p$ of them $\Rightarrow$ $p \times N^2$

                  \item The covariance matrix $D$ contains $N$ parameters (Note that the covariance matrix $D$ is diagonal)
              \end{enumerate}

              In our specific case the number of parameters is then
              \[ 4 \times 2 + 4 + 4^2 \times 2 + 4 = 48. \]
          \end{sol}
\end{enumerate}


% % Identification in SVAR models
% \subsection{Identification in \texorpdfstring{\svar{}}{SVAR} models\texorpdfstring{%
%         \protect\footnote{Based on \cite[][see pages 332-334]{Hamilton-1994} and \cite[][see chapter 14]{MartinHurnHarris-2012}.}%
%     }{}}
% Consider a \varp[1]{} model consisting of the logarithm of output and the logarithm of prices:
\[ B_0 y_t = B_1 y_{t-1} + u_t, \quad u_t \sim N(0, D) \]
where
\begin{align*}
    y_t & = \begin{pmatrix} \Delta \ln o_t \\ \Delta \ln p_t \end{pmatrix}
    \\
    B_0 & = \begin{bmatrix}
                1           & 0 \\
                b_{0, 2, 1} & 1
            \end{bmatrix}
    \\
    B_1 & = \begin{bmatrix}
                b_{1, 1, 1} & b_{1, 1, 2} \\
                b_{1, 2, 1} & b_{1, 2, 2}
            \end{bmatrix}
    \\
    D   & = \begin{bmatrix} \sigma_o^2 & 0 \\ 0 & \sigma_p^2 \end{bmatrix}
\end{align*}

\begin{enumerate}
    \item What does the imposition of $b_{0, 1, 2} = 0$ mean?
%
%          \begin{sol}
%              In the short-run, changes in the log prices does not affect the changes in the log output.
%          \end{sol}

    \item Is the \emph{order condition} satisfied?

%          \begin{sol}
%              \begin{definition}
%                  \emph{Order condition:} $B_0$ and $D$ have no more unknown parameters than V.
%              \end{definition}
%
%              \begin{enumerate}[label=$\bullet$]
%                  \item $B_0$ has one unknown parameters
%
%                  \item $D$ has two unknown parameters
%
%                  \item $V$ has three distinct unknown parameters, as it is symmetric matrix
%              \end{enumerate}
%              Therefore, the \emph{order condition} is satisfied.
%          \end{sol}

    \item Determine the following
          \begin{enumerate}[label=\roman*.]
              \item $\theta_B$
              \item $\theta_D$.
              \item $V = \E[v_t v_t'] = (B_0^{-1})' D B_0^{-1}$
              \item $vech(V)$
              \item $J$.
          \end{enumerate}


%          \begin{sol}
%              \begin{flalign*}
%                  \theta_B & = b_{2, 1, 1}
%                  \\[0.8cm] % ======
%                  \theta_D & = \begin{pmatrix}
%                                   \sigma_o^2 \\ \sigma_p^2
%                               \end{pmatrix}
%                  \\[0.8cm] % ======
%                  B_0      & = \begin{bmatrix}
%                                   1           & 0 \\
%                                   b_{0, 2, 1} & 1
%                               \end{bmatrix}
%                  \\[0.8cm] % ======
%                  B_0^{-1} & = \begin{bmatrix}
%                                   1            & 0 \\
%                                   -b_{0, 2, 1} & 1
%                               \end{bmatrix}
%                  \\[0.8cm] % ======
%                  V        & = B_0^{-1} D (B_0^{-1})'
%                  \\[0.8cm] % ==========
%                           & = \begin{bmatrix}
%                                   1            & 0 \\
%                                   -b_{0, 2, 1} & 1
%                               \end{bmatrix}
%                  \begin{bmatrix}
%                      \sigma_o^2 & 0          \\
%                      0          & \sigma_p^2
%                  \end{bmatrix}
%                  \begin{bmatrix}
%                      1 & -b_{0, 2, 1} \\
%                      0 & 1
%                  \end{bmatrix}
%                  \\[0.8cm] % ==========
%                           & = \begin{bmatrix}
%                                   \sigma_o^2
%                                    & 0          \\
%                                   -b_{0, 2, 1} \sigma_o^2
%                                    & \sigma_p^2
%                               \end{bmatrix}
%                  \begin{bmatrix}
%                      1 & -b_{0, 2, 1} \\
%                      0 & 1
%                  \end{bmatrix}
%                  \\[0.8cm]
%                           & = \begin{bmatrix}
%                                   \sigma_o^2
%                                    & - b_{0, 2, 1} \sigma_o^2               \\
%                                   -b_{0, 2, 1} \sigma_o^2
%                                    & -b_{0, 2, 1}^2 \sigma_o^2 + \sigma_p^2
%                               \end{bmatrix}
%                  \\[0.8cm] % ==========
%                  vech(V)  & = \begin{pmatrix}
%                                   \sigma_o^2              \\
%                                   -b_{0, 2, 1} \sigma_o^2 \\
%                                   -b_{0, 2, 1}^2 \sigma_o^2 + \sigma_p^2
%                               \end{pmatrix}
%                  \\[0.8cm] % ==========
%                  J        & = \begin{bmatrix}
%                                   \frac{\partial vech(V)}{\partial \theta_B'}
%                                    & \frac{\partial vech(V)}{\partial \theta_D'}
%                               \end{bmatrix}
%                  \\[0.8cm] % ========
%                           & = \begin{bmatrix}
%                                   0
%                                    & 1
%                                    & 0
%                                   \\
%                                   -\sigma_o^2
%                                    & -b_{0, 2, 1}
%                                    & 0
%                                   \\
%                                   -2 b_{0, 2, 1} \sigma_o^2
%                                    & -b_{0, 2, 1}^2
%                                    & 1
%                                   \\
%                               \end{bmatrix}
%              \end{flalign*}
%          \end{sol}

    \item Using \texttt{R}, check whether the \emph{rank condition} is satisfied at the following parameter values?

          \begin{enumerate}[label=\roman*.]
              \item when $b_{0, 2, 1} = 0.1$, $\sigma_o^2 = 1$, and $\sigma_p^2 = 0.1$

              \item when $b_{0, 2, 1} = 2.1$, $\sigma_o^2 = 0$, and $\sigma_p^2 = 3.1$
          \end{enumerate}
%
%          \begin{sol}
%              \begin{definition}
%                  The rank condition requires that the columns of $J$ are linearly independent.
%              \end{definition}
%              \lstinputlisting{../R-files/rank_of_J_matrix.R}
%          \end{sol}
\end{enumerate}

\subsection{Stationarity in univariate case\texorpdfstring{%
    \protect\footnote{For more details, see \cite[][see Chapter 3, Stationary ARMA processes]{Hamilton-1994}}%
}{}}
\begin{enumerate}
    \item Show that an \maq[1]{} process, $y_t = \nu_t + \psi \nu_{t-1}$ with $\nu_t \sim N(0, 1)$, is covariance-stationary.

          \begin{sol}
              We have to check the three conditions from the definition of stationarity, given in the lecture.

              \textbf{Condition 1}, $\E[y_t] = \mu < \infty$:
              \begin{align*}
                  \E[y_t] = \E[\nu_t] + \psi \E[\nu_{t-1}]
                  = 0 + \phi \times 0
                  = 0
              \end{align*}
              Thus, the mean is constant and finite.

              \textbf{Condition 2}, $\Var(y_t) = \sigma^2 < \infty$:
              \begin{align*}
                  \Var(y_t)
                   & = \Var(\nu_t) + \psi^2 \Var(\nu_{t-1})
                  = 1 + \psi^2
              \end{align*}
              Thus, the variance is constant and finite.

              \textbf{Condition 3}, $\Cov(y_t, y_{t-k}) = \gamma_k, k>0$:
              \begin{align*}
                  \Cov(y_t, y_{t-k})
                   & = \Cov(\nu_t + \psi \nu_{t-1}, \nu_{t-k} + \psi \nu_{t-k-1})
                  \\
                   & = \Cov(\nu_t, \nu_{t-k})
                  + \Cov(\nu_t, \nu_{t-k-1})
                  + \Cov(\nu_{t-1}, \nu_{t-k})
                  + \Cov(\nu_{t-1}, \nu_{t-k-1})
                  \\
                   & \Downarrow
                  \\
                  \text{For $k=1$}: \quad \Cov(y_t, y_{t-1})
                   & = 0 + 0 + \Cov(\nu_{t-1}, \nu_{t-1}) + 0 = 1
                  \\
                  \text{For $k=2$}: \quad \Cov(y_t, y_{t-2})
                   & = 0 + 0 + 0 + 0 = 0
                  \\
                  \text{For $k=3$}: \quad \Cov(y_t, y_{t-3})
                   & = 0 + 0 + 0 + 0 = 0
                  \\
                   & \vdots
              \end{align*}
              Thus, the covariance between $y_t$ and $y_{t-k}$ is only a function of the time between two points, and not of time itself.
          \end{sol}

    \item Using R, simulate $2500$ observations from the \arp[1]{} process, $y_t = \phi y_{t-1} + \nu_t$ with $\nu_t \sim N(0, 1)$, for each $\phi = 0.99$, $\phi = 1.00$, and $\phi = 1.01$. Plot the simulated series and the corresponding auto-correlation function (ACF). Try and guess which cases are stationary and which cases are non-stationary. 

    \begin{sol}
        \lstinputlisting{../R-files/simulate_ar1.R}
    \end{sol}

    \item Consider an \armapq{} process, $\Phi(L) X_t = \Theta(L) \nu_t$:
        \begin{enumerate}[label=$\bullet$]
        \item \armapq{} processes are \emph{stable}, if all roots of 
        \[ \Phi(z) = 0 \]
        are larger than 1 in absolute values.

        \item \armapq{} processes are \emph{invertible}, if all roots of 
        \[ \Theta(z) = 0 \]
        are larger than 1 in absolute values.\footnote{For more details, see \cite[][Chapter 3, "Stationary ARMA Processes"]{Hamilton-1994}}
        \end{enumerate}

        Check whether the following \armapq[2, 2]{} is stable and invertible.
        \[
            X_t
            = 0.2 X_{t-1} + 0.48 X_{t-2}
            + \nu_t - 0.2 \nu_{t-1} - 0.08 \nu_{t-2}
        \]

        \begin{sol}
            \begin{align*}
                X_t
                & = 0.2 X_{t-1} + 0.48 X_{t-2}
                + \nu_t - 0.2 \nu_{t-1} - 0.08 \nu_{t-2}
                \\
                X_t - 0.2 X_{t-1} - 0.48 X_{t-2} 
                & = \nu_t - 0.2 \nu_{t-1} - 0.08 \nu_{t-2}
                \\
                (1 - 0.2 L - 0.48 L^2) X_t 
                & = (1 - 0.2 L - 0.08 L^2) \nu_t
            \end{align*}

            Using the following R script, we find that both $\Phi(z) = 0$ and $\Theta(z) = 0$ have roots outside the unit circle.
            \lstinputlisting{../R-files/polynomial_roots.R}
            Thus, the process is both stable and invertible.
        \end{sol}
\end{enumerate}

\subsection{Stationarity in multivariate case}
\begin{enumerate}
    \item How does the definition of covariance stationarity change in the multivariate case?\footnote{Taken from \cite[][See section 16, "Understanding multivariate time series concepts"]{Mutschler-2018-github_repo}.}

          \begin{sol}
              Basically, it is the same: A stochastic process is called weakly stationary (or covariance stationary), if for each period in time, it has the same expectation and variance, independent of time, and the autocovariance between two points in time is only dependent on the distance of these two points. In the multivariate case, we now also consider the autocovariances between different variables and require these to be only dependent on the distance in time, not on time itself.

              \begin{definition}
                  The K-dimensional stochastic process $y_t$ is called covariance stationary, if for all $h,t,\tau \in \mathbb{Z}$:
                  \begin{align}
                      \E[{y_t}]
                       & = \E[{y_\tau}]
                      \\
                      \Var(y_t) 
                      & = \Var(y_\tau)
                      \\
                      \Cov(y_{t},y_{t-h})
                       & = \Cov(y_{t+\tau}, y_{t-h+\tau})
                  \end{align}
              \end{definition}
          \end{sol}

    \item Interpret $\E[y_t]$ and $\Gamma_h := \Cov(y_{t},y_{t-h})$\footnote{Taken from \cite[][See section 16, "Understanding multivariate time series concepts"]{Mutschler-2018-github_repo}.}.
          \begin{sol}
              \emph{Interpretation of $\E[y_t]$:} Each component of $y_t$ has its own expectation. $\E[y_t]$ is the unconditional expectation of $y_t$ at time point $t$. The expectation as a linear operator can go into a vector or matrix:
              \begin{align*}
                  \E[y_t]
                  = \E\left[
                      \begin{pmatrix}
                          y_{1,t} \cr \vdots \cr y_{K,t}
                      \end{pmatrix}
                      \right]
                  = \begin{pmatrix}
                        \E[y_{1,t}] \cr \vdots \cr \E[y_{K,t}]
                    \end{pmatrix}
              \end{align*}


              \emph{Interpretation of $\Gamma_y(h)$:} In the univariate case, we define the autocovariance as the covariance of a random variable with its own lagged values. In the multivariate case, we also consider covariances between different variables. To this end, we want to summarize the covariance between different variables and different points in time. The Autocovariance matrix \[ {\Gamma_y(h)} := \Cov[{y_{t},{y_{t-h}}}] \] sums this information up in a neat fashion and is therefore a powerful tool in multivariate time series analysis:
              \begin{align*}
                  \Cov[{y_{t}, {y_{t-h}}}]
                   & = \E\left[
                      ({y_{t}} - \E[{{y_{t}}}])
                      ({y_{t-h}} - \E[{{y_{t-h}}}])'
                      \right]
                  \\[0.5cm]
                   & = \E\left[
                      \begin{pmatrix}
                          y_{1,t} - \E[y_{1,t}]
                          \cr \vdots
                          \cr y_{K,t} - \E[y_{K,t}]
                      \end{pmatrix}
                      \begin{pmatrix}
                          y_{1,{t-h}} - \E[y_{1,{t-h}}]
                           & \dots
                           & y_{K,{t-h}} - \E[y_{K,{t-h}}]
                      \end{pmatrix}
                      \right]
                  \\[0.5cm]
                   & = \E\left[
                      \begin{pmatrix}
                          (y_{1,t}- \E[y_{1,t}])(y_{1,{t-h}} - \E[y_{1,{t-h}}])
                           & \dots
                           & (y_{1,t} - \E[y_{1,t}])(y_{K,{t-h}} - \E[y_{K,{t-h}}])
                          \\
                          \vdots
                           & \ddots
                           & \vdots
                          \\
                          (y_{K,t}- \E[y_{K,t}])(y_{1,{t-h}} - \E[y_{1,{t-h}}])
                           & \dots
                           & (y_{K,t} - \E[y_{K,t}])(y_{K,{t-h}} - \E[y_{K,{t-h}}])
                      \end{pmatrix}
                      \right]
                  \\[0.5cm]
                   & = \begin{pmatrix}
                           \E[(y_{1,t}- \E[y_{1,t}])(y_{1,{t-h}} - \E[y_{1,{t-h}}])]
                            & \dots
                            & \E[(y_{1,t} - \E[y_{1,t}])(y_{K,{t-h}} - \E[y_{K,{t-h}}])]
                           \\
                           \vdots
                            & \ddots
                            & \vdots
                           \\
                           \E[(y_{K,t}- \E[y_{K,t}])(y_{1,{t-h}} - \E[y_{1,{t-h}}])]
                            & \dots
                            & \E[(y_{K,t} - \E[y_{K,t}])(y_{K,{t-h}} - \E[y_{K,{t-h}}])]
                       \end{pmatrix}
                  \\[0.5cm]
                   & =\begin{pmatrix}
                          \Cov(y_{1,t},y_{1,{t-h}})
                           & \dots
                           & \Cov(y_{1,t},y_{K,{t-h}})
                          \\
                          \vdots
                           & \ddots
                           & \vdots
                          \\
                          \Cov(y_{K,t},y_{1,{t-h}})
                           & \dots
                           & \Cov(y_{K,t},y_{K,{t-h}})
                      \end{pmatrix}
              \end{align*}
              On the diagonals, we have the autocovariance of each variable, on the off-diagonals we have the covariances between the different variables.
          \end{sol}

    \item Show that the following \varp[2]{} process is covariance stationary.

    \begin{sol}
        
    \end{sol}
\end{enumerate}


\subsection{Theoretical and simulated moments\texorpdfstring{%
    \protect\footnote{Taken from \cite[][see Sectiona 18, "Theoretical and simulated moments]{Mutschler-2018-github_repo}}%
}{}}
\begin{enumerate}
\item Derive the theoretical mean ($\mu_y$), covariance matrix ($\Gamma(0)$) and autocovariance matrix ($\Gamma(h)$) of the covariance stationary \varp[1]{} model: $y_t = \mu + \Phi y_{t-1} + \nu_t, \quad \nu_t \sim N(0, V)$

\begin{sol}
    \textbf{Mean:} \\
    \begin{align*}
        \underbrace{\E(y_t)}_{\mu_y}
        & = \underbrace{\E(\mu)}_{\mu}
        + \underbrace{\E(\Phi y_{t-1})}_{\Phi \mu_y}
        + \underbrace{\E(\nu_t)}_{0}
        \\
        \Leftrightarrow \mu_y      
        & = (I - \Phi)^{-1} \mu
    \end{align*}
    Put $\mu = (I - \Phi) \mu_y:
    \underbrace{y_t -\mu_y}_{ \tilde{y}_{t}}
    = \Phi (\underbrace{y_{t-1} - \mu_y}_{\tilde{y}_{t-1}})
    + u_t$.


    \textbf{Covariance matrix and Autocovariance matrix:} \\
    Note: $\Gamma_y(h) = \Gamma_{\tilde{y}}(h)$
    \begin{align*}
        \Gamma_y(0)
        & = \E\left[\tilde{y}_t \tilde{y}_t'\right]
        = \E\left[(\Phi \tilde{y}_{t-1} + \nu_t)(\Phi \tilde{y}_{t-1} + \nu_t)'\right]
        \\
        & = \Phi \underbrace{\E\left[\tilde{y}_{t-1}\right]}_{\Gamma_y(0)} \Phi'
        + \Phi \underbrace{\E\left[\tilde{y}_{t-1}\nu_t'\right]}_{0}
        + \underbrace{\E\left[\nu_t \tilde{y}_{t-1}'\right]}_{0} \Phi'
        + \underbrace{\E\left[\nu_t \nu_t'\right]}_{V}
        \\
        \Leftrightarrow \Gamma_y(0)
        & = \Phi \Gamma_y(0) \Phi' + V
    \end{align*}
    This is a discrete Lyapunov matrix equation, see exercise 1.5:
    \[
        vec\bigg( \Gamma_y(0) \bigg)
        = \bigg( I - \Phi \otimes \Phi \bigg)^{-1} vec\bigg( \Sigma_u \bigg)
    \]
    \begin{align*}
        \Gamma_y(1)
        & = \E\left[\tilde{y}_t \tilde{y}_{t-1}'\right]
        = \E\left[(\Phi \tilde{y}_{t-1} + \nu_t)\tilde{y}_{t-1}'\right]
        = \Phi \E\underbrace{
            \left[\tilde{y}_{t-1}\tilde{y}_{t-1}'\right]
        }_{\Gamma_y(0)}
        + \underbrace{\E\left[\nu_t y_{t-1}'\right]}_{0}
        = \Phi \Gamma_y(0)
        \\
        \Gamma_y(2)
        & =  \E\left[\tilde{y}_t \tilde{y}_{t-2}'\right]
        = \E\left[(\Phi \tilde{y}_{t-1} + \nu_t)\tilde{y}_{t-2}'\right]
        = \Phi \Gamma_y(1) = \Phi^2 \Gamma_y(0)
        \\
        \Gamma_y(h)
        & = \Phi^h \Gamma_y(0)
    \end{align*}
\end{sol}

\item Simulate $R = 100$ datasets each with $T = 100$ observations for
\begin{align*}
    y_t = \begin{pmatrix}0.2 &0.3 \\-0.6 & 1.1 \end{pmatrix} y_{t-1}  + \nu_t
\end{align*}
provided that $\nu_t \sim N(0, V)$ and $V = \begin{pmatrix}
    0.9 & 0.2 \\ 0.2 & 0.5
\end{pmatrix}$.

Compute the sample mean and sample covariance matrix. Compare the results with parts (1) and (2). How does your choice of $R$ or $T$ change results?

\begin{sol}
  \lstinputlisting{../R-files/var1_simulate_moments.R}
    The Monte Carlo shows that either increasing $R$ or $T$ (or both) captures the moments better.
\end{sol}
\end{enumerate}

\subsection{Transforming a VAR to a VMA}
\begin{enumerate}
    \item Using R, plot the Auto-correlation function (ACF) and the Partial Auto-correlation function (PACF) of the follwoing \ar{} and \ma{} models.

    \begin{sol}
        
    \end{sol}

    \item Finding the coefficients in the univariate case

    \begin{sol}

    \end{sol}

    \item Consider the following VAR(1) process $y_t = \mu + \Phi_1 y_{t-1} + \nu_t$
          \begin{align*}
              \begin{pmatrix} y_{1,t} \\ y_{2,t} \\ y_{3,t} \end{pmatrix}
               & = \begin{pmatrix} 0 \\ 0 \\ 0 \end{pmatrix}
              + \begin{pmatrix}
                    0.5 & 0   & 0   \\
                    0.1 & 0.1 & 0.3 \\
                    0   & 0.2 & 0.3
                \end{pmatrix}
              \begin{pmatrix}
                  y_{1, {t-1}} \\ y_{2, {t-1}} \\ y_{3, {t-1}}
              \end{pmatrix}
              + \begin{pmatrix}
                    \nu_{1,t} \\ \nu_{2,t} \\ \nu_{3,t}
                \end{pmatrix}
          \end{align*}
          provided that $\nu_t \sim N(0, V)$ and
          \[
              V
              = \begin{pmatrix}
                  2.25 & 0 & 0 \\ 0 & 1 & 0.5 \\ 0 & 0.5 & 0.74
              \end{pmatrix}
          \]
          Compute the coefficients $\Psi_0, \Psi_1, \dots \in \mathbb{R}^{3\times 3}$ of the lag polynomial $\Psi(L) := \sum_{i=0}^\infty \Psi_i L^i$, and a $c \in \mathbb{R}^3$ such that\footnote{Taken from \cite[][See section 16, "Understanding multivariate time series concepts"]{Mutschler-2018-github_repo}.}
          \begin{align*}
              y_t = c + \Psi(L) \nu_t
          \end{align*}

          \begin{sol}
              We will use the very efficient method of matching coefficients to transform the \varp[1]{} model into a \vmaq[$\infty$]{} representation.
              \begin{align*}
                  y_t
                   & = \Phi_1 y_{t-1} + \nu_t
                  \\
                  \underbrace{(I_3 - \Phi_1 L)}_{\Phi(L)} y_t
                   & = \nu_t
                  \\
                  \Phi(L) y_t
                   & = \nu_t
                  \\
                  \Phi(L)^{-1} \Phi(L) y_t
                   & = \Phi(L)^{-1} \nu_t
                  \\
                  y_t
                   & = \Phi(L)^{-1} \nu_t
                   = c + \Psi(L) \nu_t
                  \\
                  \Rightarrow
                   & c = 0,
                   \Psi(L) = \Phi(L)^{-1}
              \end{align*}
              In the method of matching coefficient we compare the coefficient matrices multiplied to each power of $L$. That is, the expression on the left hand side has to match the expression on the right hand side. In our case:
              \begin{align*}
                  \Psi(L)
                   & = \Phi(L)^{-1}
                  \\
                  \Phi(L) \Psi(L)
                   & = I_3
                  \\
                  (I_3 - \Phi_1 L) \left(\sum_{i=0}^\infty \Psi_i L^i\right)
                   & = I_3
              \end{align*}

              Expanding the two brackets:
              \begin{align*}
                  \Psi_0 L^0
                   & + \Psi_1 L^1 + \Psi_2 L^2 + \dots
                  \\
                   & - \Phi_1 \Psi_0 L^1 - \Phi_1 \Psi_1 L^2 - \dots = I_3 L^0
              \end{align*}

              Finally, let us compare the expressions on the left hand side to those on the right hand side:
              \begin{align}
                  L^0
                   & : \Phi_0 = I_3
                   \Rightarrow
                   \Phi_0 = I_3
                  = \begin{pmatrix}
                        1 & 0 & 0 \\ 0 & 1 & 0 \\ 0 & 0 & 1
                    \end{pmatrix}
                    \nonumber \\
                  L^1
                   & : \Phi_1 - A \Phi_0 = 0
                   \Rightarrow
                   \Phi_1 = A \Phi_0
                  = A
                  = \begin{pmatrix}
                        0.5 & 0 & 0 \\ 0.1 & 0.1 & 0.3 \\ 0 & 0.2 & 0.3
                    \end{pmatrix}
                  \nonumber \\
                  L^2
                   & : \Phi_2 - A \Phi_1 = 0 
                   \Rightarrow
                   \Phi_2 = A \Phi_1
                  = A^2
                  = \begin{pmatrix}
                        0.25 & 0 & 0 \\ 0.06 & 0.07 & 0.12 \\ 0.02 & 0.08 & 0.15
                    \end{pmatrix}
                  \nonumber \\
                  & \vdots
                  \nonumber \\
                  \text{In general} & : \Phi_s = A^s \label{allg}
              \end{align}
          \end{sol}
\end{enumerate}

% \subsection{Estimation of VAR models}
% Give a var 1 process

\begin{enumerate}
    \item Derive the log-likelihood function

    \item Numerical optimization
\end{enumerate}

\end{enumerate}

% Information criteria and Hypothesis tests

% Granger Causality

% IRFs

% Variance Decomposition

% SVAR: short-run identification

% SVAR: long-run identification

% SVAR: short- and long-run identification

% SVAR: sign restrictions identification

\Closesolutionfile{ans}

\end{document}
