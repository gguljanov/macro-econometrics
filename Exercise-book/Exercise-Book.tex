\documentclass[11pt]{article}

% === Packages ===
\usepackage[a-1b]{pdfx}

\usepackage[utf8]{inputenc}

\usepackage{amssymb}
\usepackage{amsmath}
\usepackage{amsthm}
\newtheorem{definition}{Definition}[section]

\usepackage{amsfonts}

\usepackage{nicefrac}

\usepackage{graphicx}

\usepackage{multirow}

% For using \toprule and \midrule in the tables
\usepackage{booktabs}

\usepackage{caption}

\usepackage{dcolumn}

\usepackage{siunitx}

% Change the title format
\usepackage[explicit]{titlesec}
\titleformat{\section}{\normalfont\Large\bfseries}{}{0em}{%
    Ex. \thesection ~ -- ~ #1%
}
% \titleformat{\subsection}{\normalfont\large\bfseries}{}{0em}{%
%     Ex. \thesubsection ~ -- ~ #1%
% }

% Enumerating items
\usepackage{enumitem}
\setlist[enumerate]{font=\bfseries, label=(\alph*)} %, itemsep=0.5cm}

\usepackage{scrextend}

% Set bookmarks for sections, equations etc.
% \usepackage[bookmarks, bookmarksopen, bookmarksdepth=3]{hyperref}
\hypersetup{bookmarksnumbered}

% % Different colors for font and the background
% \usepackage[dvipsnames]{xcolor}

% Cancel by striking through
\usepackage{cancel}

% Set the space between lines
\usepackage{setspace}

% Include code chunks beautifully
\usepackage{listingsutf8}
\lstset{
    language=R,
    frame=tb,
    basicstyle=\ttfamily\footnotesize,
    alsoletter={.},
    breaklines=true,
    numbers=left,
    numberstyle=\tiny\color{gray},
    showstringspaces=false
}

% References
\usepackage[citestyle=alphabetic, bibstyle=authortitle]{biblatex}
\addbibresource{biblio.bib}

% \today and \currentime commands
\usepackage{datetime}

% \usepackage{todonotes}
% \usepackage{tikz}

% Space between lines
\usepackage{setspace}
\linespread{1.2}

% Page margins
\usepackage[top=2.5cm, bottom=2.5cm, left=2.5cm, right=2cm]{geometry}

% Space between paragraphs
\usepackage{parskip}

% Paragraph indentation
\setlength{\parindent}{0pt}


% === Commands ===
% Expectation operator
\DeclareMathOperator*{\E}{\mathbb{E}}
\DeclareMathOperator*{\Var}{\mathbb{V}ar}
\DeclareMathOperator*{\Cov}{\mathbb{C}ov}

% Colored cancel
\newcommand*{\Ccancel}[2][black]{
    \renewcommand{\CancelColor}{\color{#1}}{\cancel{#2}}
}

\newtheorem*{definition*}{Definition}

% Writing ß with \3
\let\3 = \ss

\newcommand*{\iid}{\overset{\text{iid}}{\sim}}

% Writing code and output
\newcommand*{\writeCode}{
    \vspace{0.3cm}
    \textcolor{black}{\underline{\textbf{Code}}}:
}

\newcommand*{\writeOutput}{
    \vspace{0.3cm}
    \textcolor{black}{\underline{\textbf{Output}}}:
    \vspace{-0.8cm}
}


% === Others ===
% Allows box-overflow amount before a warning message is issued
\hfuzz=6000pt

% Allows to control the space added before the math equations
\abovedisplayskip=-0.2cm
\abovedisplayshortskip=-0.2cm

% Allows to control the space added after the math equations
\belowdisplayskip=-0.2cm
\belowdisplayshortskip=-0.2cm

% Allows the align math expression be broken into pages at the output pdf
\allowdisplaybreaks

% Check mark and X mark
\usepackage{pifont}

\newcommand{\cmark}{\ding{51}}
\newcommand{\xmark}{\ding{55}}

% Wide hat
\usepackage{scalerel, stackengine}
\stackMath

\newcommand\reallywidehat[1]{%
    \savestack{\tmpbox}{\stretchto{%
            \scaleto{%
                \scalerel*[\widthof{\ensuremath{#1}}]{%
                    \kern-.6pt\bigwedge\kern-.6pt%
                }{
                    \rule[-\textheight/2]{1ex}{\textheight}
                }%WIDTH-LIMITED BIG WEDGE
            }{\textheight}% 
        }{1ex}}%
    \stackon[1pt]{#1}{\tmpbox}%
}

%  === Abbreviations and Acronyms ===
\newcommand{\ols}{Ordinary Least Squares}
\newcommand{\ml}{Maximum Likelihood}

\newcommand*{\ar}{AR}
\newcommand*{\arp}[1][p]{AR(#1)}

\newcommand*{\var}{VAR}
\newcommand*{\varp}[1][p]{VAR(#1)}

\newcommand*{\svar}{SVAR}
\newcommand*{\svarp}[1][p]{SVAR(#1)}

\newcommand*{\ma}{MA}
\newcommand*{\maq}[1][q]{MA(#1)}

\newcommand*{\vma}{VMA}
\newcommand*{\vmaq}[1][q]{VMA(#1)}

\newcommand*{\arma}{ARMA}
\newcommand*{\armapq}[1][p, q]{ARMA(#1)}

\newcommand*{\varma}{VARMA}
\newcommand*{\varmapq}[1][p, q]{VARMA(#1)}

\newcommand*{\arima}{ARIMA}
\newcommand*{\arimapdq}[1][p, d, q]{ARIMA(#1)}

\newcommand*{\arch}{ARCH}
\newcommand*{\archp}[1][p]{ARCH(#1)}

\newcommand*{\garch}{GARCH}
\newcommand*{\garchpq}[1][p, q]{GARCH(#1)}


% === Solutions open/closed ===
% \usepackage{answers}
\usepackage[nosolutionfiles]{answers}

\renewcommand{\solutionstyle}[1]{\textbf{Sol. #1}:\hspace*{\textwidth}}
\newcommand\preSolution{\vspace*{0.2cm}}
\newcommand\postSolution{\begin{flushright}\textbf{End of solution}\end{flushright}\vspace*{0.3cm}}

\Newassociation{sol}{Solution}{ans}

% Ouput only a part of the book
% \usepackage[files, -21]{pagesel}

% === Font ===
\usepackage{newpxtext, newpxmath}


% === Document begins ===
\begin{document}

\Opensolutionfile{ans}[ans]

\title{Macro-Econometrics --- Exercise Book}

\author{}

\date{This version: \today ~\currenttime}

\maketitle

% === References ===
The references for this tutorial are
\begin{enumerate}[label=\roman*.]
    \item \cite{MartinHurnHarris-2012}
    \item \cite{Mutschler-2018-github_repo}
    \item Other explicit citations within the text
\end{enumerate}

% === Exercises ===
% Notes about matrices
\section{Notes about matrices\texorpdfstring{%
      \protect\footnote{Based mainly on \cite{MagnusNeudecker-2019-part1} and \cite{MagnusNeudecker-2019-part2}; some parts are from \cite[][]{Hamilton-1994} and \cite[][]{Mutschler-2018-github_repo}}%
  }{}}
\begin{labeling}{argument}
    \item[\textbf{Matrix:}] A rectangular array of real numbers:
    \[
        \begin{bmatrix}
            a_{11} & a_{12} & \ldots a_{1n} \\
            a_{21} & a_{22} & \ldots a_{2n} \\
            \vdots & \vdots & \vdots        \\
            a_{m1} & a_{m2} & \ldots a_{mn}
        \end{bmatrix}
    \]

    For the following definitions, we use $m \times n$ matrices $A$ and $B$. When we talk about square matrices, we assume $m = n$.

    \item[\textbf{Addition:}] Adding two matrices is adding the respective elements:
    \[ A + B = (a_{ij}) + (b_{ij}) = (a_{ij} + b_{ij}). \]
    Note that the dimensions of the summands must be the same.

    \textbf{Exercise:} Do the following addition:
    \[
        \begin{bmatrix}
            1  & 12 & 8  \\
            15 & 6  & 10
        \end{bmatrix}
        +
        \begin{bmatrix}
            3   & 2 & 16  \\
            -14 & 7 & -11
        \end{bmatrix}
        =
        ?
    \]

    \begin{sol}
        \[
            \begin{bmatrix}
                4 & 14 & 24 \\
                1 & 13 & -1
            \end{bmatrix}
        \]
    \end{sol}


    \item[\textbf{Multiplication:}] Multiply each row of the first matrix with each column of the second matrix, element-wise and take sum. Order of the row and column gives the place of the sum in the resulting matrix:
    \[
        AB = \bigg( \sum_{j=1}^n a_{ij} b_{jk} \bigg)
    \]

    \textbf{Exercise:} Do the following multiplication:
    \[
        \begin{bmatrix}
            2  & -7 \\
            -9 & 5  \\
            4  & 3
        \end{bmatrix}
        \times
        \begin{bmatrix}
            -6 & 1 \\
            3  & 8
        \end{bmatrix}
        =
        ?
    \]

    \begin{sol}
        \[
            \begin{bmatrix}
                2 \times (-6) + (-7) \times 3
                 & 2 \times 1 + (-7) \times 8 \\
                -9 \times (-6) + 5 \times 3
                 & -9 \times 1 + 5 \times 8   \\
                4 \times (-6) + 3 \times 3
                 & 4 \times 1 + 3 \times 8
            \end{bmatrix}
            =
            \begin{pmatrix}
                -33 & -54 \\
                69  & 31  \\
                -15 & 28
            \end{pmatrix}
        \]
    \end{sol}

    \item[\textbf{Transpose:}] Turning rows into columns or vice versa:
    \[
        (a'_{ij}) = (a_{ji})
    \]
    It is denoted as $A'$.

    \textbf{Exercise:} Take the transpose of the following matrix

    \[
        A = \begin{bmatrix}
            2  & -7 \\
            -9 & 5  \\
            4  & 3
        \end{bmatrix}
    \]

    \begin{sol}
        \[
            A' = \begin{bmatrix}
                2  & -9 & 4 \\
                -7 & 5  & 3
            \end{bmatrix}
        \]
    \end{sol}

    \item[\textbf{Square Matrix:}] The number of rows of a square matrix is equal to its number of columns.

    \textbf{Exercise:} Is the following matrix square?
    \[
        A = \begin{bmatrix}
            2  & -7 \\
            -9 & 5
        \end{bmatrix}
    \]

    \begin{sol}
        Yes, it is. Because, it has two rows and two columns.
    \end{sol}

    \item[\textbf{Rank:}] Number of linearly independent rows/columns is called row/column rank. A set of vectors $x_1, x_2, \ldots x_n$ is set to be linearly independent, if there exists no set of scalars $\alpha_1, \alpha_2 \ldots \alpha_n$, not all zero, such that $\sum \alpha_i x_i = 0$.\footnote{\cite[][see page 8]{MagnusNeudecker-2019-part1}} The rank is denoted as $r(A)$.

    \textbf{Exercise:} Find the rank of the following matrix using R:

    \[
        A = \begin{bmatrix}
            2  & -7 \\
            -9 & 5  \\
            4  & 3
        \end{bmatrix}
    \]

    \begin{sol}
        \lstinputlisting{../R-files/matrix_rank.R}

        The R-script tells that the rank of matrix $A$ is two, i.e. $r(A) = 2$. Note that the rank of a matrix can be at most $min(m, n)$, where $m$ is number of rows and $n$ is number of columns. Here, we have three rows and two columns. So, the rank can be at most two.
    \end{sol}

    \item[\textbf{Inverse:}] When you multiply a matrix with its inverse, you get an identity matrix:
    \[
        A A^{-1} = A^{-1} A = I_n
    \]
    Note that matrix $A$ must be square matrix. So, here $m = n$ for matrix $A$. Moreover, it might be that an inverse does not exist. The inverse is denoted as $A^{-1}$.

    \textbf{Exercise:} Take the inverse of the following matrix and using R, check, if your answer is correct by multiplying the inverse with the original matrix:
    \[
        A = \begin{bmatrix}
            2  & -7 \\
            -9 & 5
        \end{bmatrix}
    \]

    \begin{sol}
        For two-by-two matrices, there is a simple formula for inverting:
        \[
            \begin{bmatrix}
                -0.09433962 & -0.13207547 \\
                -0.16981132 & -0.03773585
            \end{bmatrix}
        \]
    \end{sol}

    \item[\textbf{Determinant:}] Determinant is a special number associated with a square matrix. In R, you can find the determinant of matrix $A$ with \verb|determinant(A, logarithm = FALSE)|. The determinant is denoted as $|A|$.

    \textbf{Exercise:} Find the determinant of the following matrix:
    \[
        A = \begin{bmatrix}
            1  & 12 & 8  & 13 \\
            15 & 6  & 10 & 3  \\
            14 & 7  & 11 & 2  \\
            4  & 9  & 5  & 16 \\
        \end{bmatrix}
    \]

    \begin{sol}
        \lstinputlisting{../R-files/matrix_determinant.R}
    \end{sol}

    \item[\textbf{Trace:}] Sum of a diagonal elements of a square matrix. It is denoted as $tr(A)$.

    \textbf{Exercise:} Find the trace of the following matrix.
    \[
        A = \begin{bmatrix}
            1  & 12 & 8  & 13 \\
            15 & 6  & 10 & 3  \\
            14 & 7  & 11 & 2  \\
            4  & 9  & 5  & 16 \\
        \end{bmatrix}
    \]

    \begin{sol}
        $tr(A) = 1 + 6 + 11 + 16 = 34$
    \end{sol}

    \item[\textbf{Eigenvalues:}] The eigenvalues of $A$ are defined as the roots of the characteristic equation:\footnote{This definition is from \cite[][page 12]{MagnusNeudecker-2019-part1}}
    \[ |\lambda I_n - A| = 0 \]

    \item[\textbf{Eigenvectors:}] Let $\lambda$ be an eigenvalue of $A$. Then, there exists a vector $x$ ($x \neq 0$) such that
    \[ (\lambda I - A) x = 0. \]
    The vector $x$ is called a (column) eigenvector of $A$ associated with the eigenvalue $\lambda$.\footnote{This definition is from \cite[][page 12]{MagnusNeudecker-2019-part1}}

    \textbf{Exercise:} Using R, determine the eigenvalues and eigenvectors of the following matrix:
    \[
        A = \begin{bmatrix}
            0.5 & 0   & 0   \\
            0.1 & 0.1 & 0.3 \\
            0   & 0.2 & 0.3
        \end{bmatrix}
    \]

    \begin{sol}
        \lstinputlisting{../R-files/eigen_values_vectors.R}
    \end{sol}

    \item[\textbf{Vectorization:}] Stacks the columns of a matrix one after another, vertically, resulting in a long column vector:
    \begin{align*}
        A
        = \begin{bmatrix}
              a_{11} & a_{12} & a_{13} \\
              a_{21} & a_{22} & a_{23} \\
              a_{31} & a_{32} & a_{33}
          \end{bmatrix},
        \quad \quad
        vec(A)
        = \begin{pmatrix}
              a_{11} \\ a_{21} \\ a_{31} \\
              a_{12} \\ a_{22} \\ a_{32} \\
              a_{13} \\ a_{23} \\ a_{33}
          \end{pmatrix}.
    \end{align*}

    \textbf{Exercise:} Determine the following:
    \begin{align*}
        vec \begin{bmatrix}
                1 & 3 & 2 \\
                1 & 0 & 0 \\
                1 & 2 & 2
            \end{bmatrix}
        =
        ?
    \end{align*}

    \begin{sol}
        \begin{align*}
            vec \begin{pmatrix}
                    1 & 3 & 2 \\
                    1 & 0 & 0 \\
                    1 & 2 & 2
                \end{pmatrix}
            =
            \begin{pmatrix}
                1 \\ 1 \\ 1 \\ 3 \\ 0 \\ 2 \\ 2 \\ 0 \\ 2
            \end{pmatrix}
        \end{align*}
    \end{sol}

    \item[\textbf{Half-vectorization:}] Similar to $vec(\cdot)$ operator, but considers only those elements on or below the diagonal:\footnote{for more details, see \cite[][page 300-301]{Hamilton-1994}}
    \begin{align*}
        A
        = \begin{bmatrix}
              a_{11} & a_{12} & a_{13} \\
              a_{21} & a_{22} & a_{23} \\
              a_{31} & a_{32} & a_{33}
          \end{bmatrix},
        \quad \quad
        vech(A)
        = \begin{pmatrix}
              a_{11} \\ a_{21} \\ a_{31} \\ a_{22} \\ a_{32} \\ a_{33}
          \end{pmatrix}.
    \end{align*}

    \textbf{Exercise:} Determine the following:
    \begin{align*}
        vech \begin{pmatrix}
                 1 & 3 & 2 \\
                 1 & 0 & 0 \\
                 1 & 2 & 2
             \end{pmatrix}
        =
        ?
    \end{align*}

    \begin{sol}
        \begin{align*}
            vech \begin{pmatrix}
                     1 & 3 & 2 \\
                     1 & 0 & 0 \\
                     1 & 2 & 2
                 \end{pmatrix}
            =
            \begin{pmatrix}
                1 \\ 1 \\ 1 \\ 0 \\ 2 \\ 2
            \end{pmatrix}
        \end{align*}
    \end{sol}

    \item[\textbf{Kronecker Product:}] Multiplies each element of the first matrix with the entire second matrix:
    \[
        A \otimes B = \begin{bmatrix}
            a_{11} B & \cdots & a_{1n} B \\
            \vdots   &        & \vdots   \\
            a_{m1} B & \cdots & a_{nn} B \\
        \end{bmatrix}
    \]

    \textbf{Exercise:}
    \[
        \underbrace{%
            \begin{bmatrix}
                1 & 3 & 2 \\
                1 & 0 & 0 \\
                1 & 2 & 2
            \end{bmatrix}%
        }_{3 \times 3}
        \otimes
        \underbrace{%
            \begin{bmatrix}
                0 & 5 \\
                5 & 0 \\
                1 & 1
            \end{bmatrix}%
        }_{3 \times 2}
        = ?
    \]

    \begin{sol}
        \begin{align*}
            \underbrace{%
                \begin{bmatrix}
                    1 & 3 & 2 \\
                    1 & 0 & 0 \\
                    1 & 2 & 2
                \end{bmatrix}%
            }_{3 \times 3}
            \otimes
            \underbrace{%
                \begin{bmatrix}
                    0 & 5 \\
                    5 & 0 \\
                    1 & 1
                \end{bmatrix}%
            }_{3 \times 2}
            =
            \underbrace{%
                \begin{bmatrix}
                    1\cdot\begin{pmatrix}
                              0 & 5 \\
                              5 & 0 \\
                              1 & 1
                          \end{pmatrix}
                     & 3\cdot\begin{pmatrix}
                                 0 & 5 \\
                                 5 & 0 \\
                                 1 & 1
                             \end{pmatrix}
                     & 2\cdot\begin{pmatrix}
                                 0 & 5 \\
                                 5 & 0 \\
                                 1 & 1
                             \end{pmatrix}
                    \\[0.8cm]
                    1\cdot\begin{pmatrix}
                              0 & 5 \\
                              5 & 0 \\
                              1 & 1
                          \end{pmatrix}
                     & 0\cdot\begin{pmatrix}
                                 0 & 5 \\
                                 5 & 0 \\
                                 1 & 1
                             \end{pmatrix}
                     & 0\cdot\begin{pmatrix}
                                 0 & 5 \\
                                 5 & 0 \\
                                 1 & 1
                             \end{pmatrix}
                    \\[0.8cm]
                    1\cdot\begin{pmatrix}
                              0 & 5 \\
                              5 & 0 \\
                              1 & 1
                          \end{pmatrix}
                     & 2\cdot\begin{pmatrix}
                                 0 & 5 \\
                                 5 & 0 \\
                                 1 & 1
                             \end{pmatrix}
                     & 2\cdot\begin{pmatrix}
                                 0 & 5 \\
                                 5 & 0 \\
                                 1 & 1
                             \end{pmatrix}
                \end{bmatrix}%
            }_{9 \times 6}
        \end{align*}
    \end{sol}
\end{labeling}

% EUROSTAT with r
\section{EUROSTAT with R}
Throughout the course and its tutorials, we will use data sets from Eurostat. There are basically two ways of accessing the data sets: (i) one can manually download the data sets or (ii) one can install the r package \texttt{eurostat} (\cite{LahtiHuovariKainuBiecek-2017}, \cite{eurostat-noyear}) to do it automatically. Below, we will discuss the second way of accessing the data sets, as we are planning to use it for the course.

First, we have to install the r package which can be done with the following command: \newline
\verb|install.packages("eurostat")|. Three functions from this package are very important for us:
\begin{enumerate}[label=\roman*.]
    \item \verb|get_eurostat_toc()| -- for getting the table of contents (TOC)

    \item \verb|search_eurostat()| -- for searching for data sets

          The function requires the following input arguments:
          \begin{labeling}{argument}
              \item[pattern:] Search pattern

              \item[type:] With this argument, the user can choose which types of datasets the search should return: datasets, tables, folders or all types (the default).

              \item[fixed:] Should the exact match to pattern be searched for?
          \end{labeling}

    \item \verb|get_eurostat()| -- for downloading a data set
          \begin{labeling}{argument}
              \item[id:] This is a dataset code

              \item[filters:] This argument is used to filter out data and only get those that are required. The R-script below uses filters to take the GDP of EA, DE, IT, FR; till Period 2024; Gross domestic product at market prices; Chain linked volumes, index 2010=100; seasonally and calendar adjusted.

              \item[cache:]
          \end{labeling}
\end{enumerate}

The following R-script is a small example for using these functions: \lstinputlisting{../R-files/eurostat_with_r.R}

\textbf{Further information:} The end user does not usually have to bother where original data is downloaded, as both data sources are accessed via the main download function \verb|get_eurostat()|. If only the table id is given, the whole table is downloaded from the SDMX 2.1 REST API. If any filters are given the JSON API is used instead. However, the \verb|get_eurostat_json()| function used internally is also a user-facing function so that can be used as well.

New function in the eurostat package version 4.0.0 is the \verb|get_eurostat_interactive()| function that allows users to search and download datasets with the help of interactive menus. If the user already knows which dataset they want to download, the \verb|get_eurostat_interactive()| function can also take a dataset code as a parameter, skipping the search part of the interactive menu. Below we will demonstrate the whole process from search to download to printing a citation for the dataset, utilizing several different eurostat package functions at once.

\url{https://ropengov.github.io/eurostat/articles/eurostat_tutorial.html}

\url{https://ropengov.github.io/eurostat/reference/get_eurostat.html#ref-examples}

% Intro
\section{Introduction to Macroeconometrics\texorpdfstring{%
      \protect\footnote{Taken from \cite[][see the section no 1, titled ``Macroeconometrics''.]{Mutschler-2018-github_repo}}%
  }{}}
Broadly define the term and research topics of \emph{Macroeconometrics}.

\begin{sol}
    Definition of term:
    \begin{enumerate}[label=]
        \item Combination of modern theoretical macroeconomics (the study of aggregated variables such as economic growth, unemployment and inflation by means of structural macroeconomic models) and econometric methods (the application of formal statistical methods in empirical economics).
    \end{enumerate}

    Research topics:
    \begin{enumerate}[label=\roman*.]
        \item How to identify sources of fluctuations, e.g. how important are monetary policy shocks as opposed to other shocks for movements in aggregate output? [forecast error variance decomposition]

        \item Understand propagation of shocks, e.g. what happens to aggregate hours worked over the next two years in response to a technology shock in the current quarter? [impulse response function]

        \item Forecasting the future, e.g. how will inflation and output growth rates evolve over next eight quarters. [forecasting]

        \item Predict effect of policy changes, e.g. how will output and inflation respond to an unanticipated change in nominal interest rate? [impulse response function and forecast scenarios]

        \item Structural changes in the economy, e.g. has monetary policy changed in the early 1980s, why did volatility of many macroeconomic time series drop in the mid 1980s, [historical decomposition]

        \item How much of the recession of 1982  would have deepened had monetary policymakers not responded to output growth at all. [policy counterfactual]
    \end{enumerate}
\end{sol}

% From univariate to multivariate models
\section{From univariate to multivariate models}
\begin{enumerate}
    \item Consider the following two \armapq[1, 1]{} models, for inflation and output, respectively:
          \[ \pi_t = 0.3 + 0.8 \pi_{t-1} + \epsilon_t - 0.1 \epsilon_{t-1} \]
          and
          \[ o_t = -0.64 o_{t-1} + \eta_t + 0.2 \eta_{t-1}. \]

          Write these two \armapq[1, 1]{} models as \varmapq[1, 1]{}.

          \begin{sol}
              \begin{align*}
                  \underbrace{
                      \begin{pmatrix}
                          \pi_t \\ o_t
                      \end{pmatrix}
                  }_{
                      y_t
                  }
                  = \underbrace{
                      \begin{pmatrix} 0.3 \\ 0 \end{pmatrix}
                  }_{
                      \mu
                  }
                  + \underbrace{
                      \begin{bmatrix}
                          0.8 & 0    \\
                          0   & -0.6
                      \end{bmatrix}
                  }_{
                      \Phi_1
                  }
                  \underbrace{
                      \begin{pmatrix}
                          \pi_{t-1} \\ o_{t-1}
                      \end{pmatrix}
                  }_{
                  y_{t-1}
                  }
                  + \underbrace{
                      \begin{pmatrix}
                          \epsilon_{t} \\ \eta_{t}
                      \end{pmatrix}
                  }_{
                      \nu_t
                  }
                  + \underbrace{
                      \begin{bmatrix}
                          -0.1 & 0   \\
                          0    & 0.2
                      \end{bmatrix}
                  }_{
                      \Psi_1
                  }
                  \underbrace{
                      \begin{pmatrix}
                          \epsilon_{t-1} \\ \eta_{t-1}
                      \end{pmatrix}
                  }_{
                  \nu_{t-1}
                  }
              \end{align*}
          \end{sol}

    \item Consider a \varp[1]{} model with the following parameter matrix
          \begin{align*}
              \mu    & = \begin{pmatrix} 0 \\ -0.04 \end{pmatrix}                    \\
              \Phi_1 & = \begin{bmatrix} -0.81 & 0.26 \\ -0.34 & -1.53 \end{bmatrix}
          \end{align*}

          Write the above \varp[1]{} model as two univariate models. Are the resulting univariate models \arp[1]{}?

          \begin{sol}
              Let us first write the \varmapq[1, 1]{} model more clearly:
              \begin{align*}
                  \begin{pmatrix} x_{t} \\ z_{t} \end{pmatrix}
                  = \begin{pmatrix} 0 \\ -0.04 \end{pmatrix}
                  + \begin{bmatrix} -0.81 & 0.26 \\ -0.34 & -1.53 \end{bmatrix}
                  \begin{pmatrix} x_{t-1} \\ z_{t-1} \end{pmatrix}
                  + \begin{pmatrix} \epsilon_{t} \\ \eta_{t} \end{pmatrix}
              \end{align*}

              The two univariate models would then be as follows:
              \begin{align*}
                  x_t & = -0.81 x_{t-1} + 0.26 z_{t-1} + \epsilon_t    \\
                  z_t & = -0.04 - 0.34 x_{t-1} - 1.53 z_{t-1} + \eta_t
              \end{align*}

              The univariate models above are not exactly \arp[1]{} model, but are rather, Autoregressive Distributed Lag models, because they contain not only their own lags but also lags of the other variable.
          \end{sol}

    \item Why are we concerned with multivariate time series? For example, why do we model the VAR(1) process
          \begin{align*}
              \begin{pmatrix} y_{1,t} \\ y_{2,t} \end{pmatrix}
              = \begin{pmatrix} \mu_1 \\ \mu_2 \end{pmatrix}
              + \begin{pmatrix}
                    \phi_{11} & \phi_{12} \\
                    \phi_{21} & \phi_{22}
                \end{pmatrix}
              \begin{pmatrix} y_{1, {t-1}} \\ y_{2, {t-1}} \end{pmatrix}
              + \begin{pmatrix} \nu_{1, t} \\ \nu_{2, t} \end{pmatrix}
          \end{align*}
          simultaneously, instead of two models for each variable separately?\footnote{Taken from \cite[][see the section 16, titled ``Understanding multivariate time series concepts"]{Mutschler-2018-github_repo}}

          \begin{sol}
              For the specification of multi-equation models, we require a clear distinction between exogenous and endogenous variables. In economic theory, this is often not clear or arbitrarily made in practice. Vectorautoregressive models do not need this distinction, they can rather be understood as a dynamic version of a simultaneous multi-equation model. This corresponds to reality, because economic variables are generated by dynamic processes, and often are interdependent.

              Therefore, VAR models provide a powerful instrument. They also take into account things like non-stationarity (cointegration and long-term equilibria), as well as the analysis of dynamics of random shocks / impulses.

              An Example: we are also able to consider correlations between $\nu_{1, t}$ and $\nu_{2, t}$.

              Lastly, VAR models tend to have better predictive power than multi-equation models and are often used as a benchmark for different forecasting models.
          \end{sol}

    \item Consider the following \svarp[1]{} model:
          \begin{align*}
              \underbrace{
                  \begin{bmatrix}
                      b_{0, 1, 1} & b_{0, 1, 2} \\
                      b_{0, 2, 1} & b_{0, 2, 2}
                  \end{bmatrix}
              }_{
                  B_0
              } y_t
              = \underbrace{
                  \begin{bmatrix}
                      b_{1, 1, 1} & b_{1, 1, 2} \\
                      b_{1, 2, 1} & b_{1, 2, 2}
                  \end{bmatrix}
              }_{B_1} y_{t-1}
              + u_t
          \end{align*}

          Write the above model as \varp[1]{}.

          \begin{sol}
              Let us first invert the $B_0$ matrix:
              \begin{align*}
                  B_0^{-1} & = \underbrace{
                      \frac{1}{
                          b_{0, 1, 1} b_{0, 2, 2} - b_{0, 1, 2} b_{0, 2, 1}
                      }
                  }_{:= c}
                  \begin{bmatrix}
                      b_{0, 2, 2}  & -b_{0, 1, 2} \\
                      -b_{0, 2, 1} & b_{0, 1, 1}
                  \end{bmatrix} \\
                           & =
                  \begin{bmatrix}
                      c b_{0, 2, 2}  & -c b_{0, 1, 2} \\
                      -c b_{0, 2, 1} & c b_{0, 1, 1}
                  \end{bmatrix}
              \end{align*}

              \begin{align*}
                  B_0 y_t & = B_1 y_{t-1} + u_t
                  \\[0.4cm]
                  y_t     & = B_0^{-1} B_1 y_{t-1} + B_0^{-1} u_t
                  \\[0.4cm]
                          & = \begin{bmatrix} % B_0^{-1}
                                  c b_{0, 2, 2}  & -c b_{0, 1, 2} \\
                                  -c b_{0, 2, 1} & c b_{0, 1, 1}
                              \end{bmatrix}
                  \begin{bmatrix} % B_1
                      b_{1, 1, 1} & b_{1, 1, 2} \\
                      b_{1, 2, 1} & b_{1, 2, 2}
                  \end{bmatrix}
                  y_{t-1}
                  +
                  \begin{bmatrix} % B_0^{-1}
                      c b_{0, 2, 2}  & -c b_{0, 1, 2} \\
                      -c b_{0, 2, 1} & c b_{0, 1, 1}
                  \end{bmatrix}
                  \begin{pmatrix}
                      u_{1, t} \\
                      u_{2, t}
                  \end{pmatrix}
                  \\[0.4cm]
                          & =
                  \underbrace{
                      \begin{bmatrix} % Phi_1
                          c b_{0, 2, 2} b_{1, 1, 1}
                          - c b_{0, 1, 2} b_{1, 2, 1}
                           & c b_{0, 2, 2} b_{1, 1, 2}
                          - c b_{0, 1, 2} b_{1, 2, 2}
                          \\
                          -c b_{0, 2, 1} b_{1, 1, 1}
                          + c b_{0, 1, 1} b_{1, 2, 1}
                           & -c b_{0, 2, 1} b_{1, 1, 2}
                          + c b_{0, 1, 1} b_{1, 2, 2}
                      \end{bmatrix}
                  }_{\Phi_1}
                  y_{t-1}
                  +
                  \underbrace{
                      \begin{bmatrix} % nu_t
                          c b_{0, 2, 2} u_{1, t} - c b_{0, 1, 2} u_{2, t} \\
                          -c b_{0, 2, 1} u_{1, 6} + c b_{0, 1, 1} u_{2, t}
                      \end{bmatrix}
                  }_{\nu_t}
                  \\[0.4cm]
                          & = \Phi_1 y_{t-1} + \nu_t
              \end{align*}
          \end{sol}

    \item Using R, simulate and plot $250$ observations from the following \varp[1]{} model:
          \[ y_t = \Phi_1 y_{t-1} + \nu_t \]
          where,
          \begin{align*}
              y_0    & = \begin{pmatrix}
                             0 \\
                             0
                         \end{pmatrix}    \\[0.5cm]
              \Phi_1 & = \begin{bmatrix}
                             -0.66 & 0.26  \\
                             -0.30 & -0.58
                         \end{bmatrix}    \\[0.5cm]
              \nu_t  & \sim N(0, I_{2, 2})
          \end{align*}

          \begin{sol}
              \lstinputlisting{../R-files/simulate_var1_250.R}
          \end{sol}

    \item Using R, simulate and plot $250$ observations from the following \svarp[1]{} models:
          \[ B_0 y_t = B_1 y_{t-1} + u_t \]
          where,
          \begin{align*}
              y_0 & = \begin{pmatrix}
                          0 \\
                          0
                      \end{pmatrix}    \\[0.5cm]
              B_0 & = \begin{bmatrix}
                          -0.98 & -0.42 \\
                          0.49  & 1.11
                      \end{bmatrix}    \\[0.5cm]
              B_1 & = \begin{bmatrix}
                          -0.66 & 0.26  \\
                          -0.30 & -0.58
                      \end{bmatrix}    \\[0.5cm]
              u_t & \sim N(0, I_{2, 2})
          \end{align*}

          \begin{sol}
              \lstinputlisting{../R-files/simulate_svar1_250.R}
          \end{sol}
\end{enumerate}


% Dimensions in VAR models
\section{Dimensions in \texorpdfstring{\varp{}}{VAR} models\texorpdfstring{%
      \protect\footnote{Based on \cite[][see the section 17, titled ``Dimensions and VAR(1) representation'']{Mutschler-2018-github_repo}.}%
  }{}
 }
Let $y_t$ be a $N$-dimensional covariance stationary random vector. Consider the \varp[p]{}-process
\begin{align*}
    y_t = \mu + \sum_{i=1}^p \Phi_i y_{t-i} + \nu_t, \quad \nu \sim N(0, V)
\end{align*}

\begin{enumerate}
    \item What are the dimensions of $\mu$, $\Phi_i$ and $\nu_t$?

          \begin{sol}
              $\mu$ and $u_t$ are both $N$-dimensional vectors: $\mu, \nu_t \in \mathbb{R}^{N \times 1}$.

              $\Phi_i$ is a $N \times N$-Matrix.
          \end{sol}

    \item Consider a \varp[2]{} model with $N=4$ variables and a constant term. How many parameters do we need to estimate?

          \begin{sol}
              In general, there are $N + N^2 \times p + N (N + 1) / 2$:
              \begin{enumerate}[label=-]
                  \item $\mu$ contains $N$ parameters

                  \item $\Phi_i$ contains $N^2$ parameters and there are $p$ of them $\Rightarrow$ $p \times N^2$

                  \item The covariance matrix $V$ contains $N (N + 1) / 2$ parameters (Note: since the covariance matrix is symmetric, we should consider either only upper half or only lower half of the matrix, together with the diagonal)
              \end{enumerate}

              In our specific case the number of parameters is then
              \[ 4 + 4^2 \cdot 2 + 4 \cdot (4 + 1) / 2 = 46. \]
              That's a lot! Therefore we will try to restrict some parameters (e.g. set equal to zero) or consider only small VAR systems, e.g. $N=3$ and $p=1$.
          \end{sol}
\end{enumerate}


% Dimensions in SVAR models
\section{Dimensions in \texorpdfstring{\svarp{}}{SVAR} models\texorpdfstring{%
      \protect\footnote{Based on \cite[][see the section 17, titled ``Dimensions and VAR(1) representation'']{Mutschler-2018-github_repo}.}%
  }{}}
Let $y_t$ be a $N$-dimensional covariance stationary random vector and $K$ be the number of relationships in $y_t$. Consider the \varp[p]{}-process
\begin{align*}
    B_0 y_t = \gamma + \sum_{i=1}^p B_i y_{t-i} + u_t, \quad u_t \sim N(0, D)
\end{align*}

\begin{enumerate}
    \item What are the dimensions of $\gamma$, $B_0$, $B_i$, and $u_t$?

          \begin{sol}
              $\gamma$ and $u_t$ are both $N$-dimensional vectors: $\mu, \nu_t \in \mathbb{R}^{N \times 1}$.

              $B_0$ and $B_i$ are $N \times N$-Matrices.
          \end{sol}

    \item Consider a \svarp[2]{} model with $N=4$ variables, $K=2$ relationships in $y_t$ and a constant term. How many parameters do we need to estimate?

          \begin{sol}
              In general, there are $NK + N + N^2 \times p + N$:
              \begin{enumerate}[label=-]
                  \item $B_0$ contains $N \times K$ parameters

                  \item $\gamma$ contains $N$ parameters

                  \item $B_i$ contains $N^2$ parameters and there are $p$ of them $\Rightarrow$ $p \times N^2$

                  \item The covariance matrix $D$ contains $N$ parameters (Note that the covariance matrix $D$ is diagonal)
              \end{enumerate}

              In our specific case the number of parameters is then
              \[ 4 \times 2 + 4 + 4^2 \times 2 + 4 = 48. \]
          \end{sol}
\end{enumerate}


% Identification in SVAR models
\section{Identification in \texorpdfstring{\svar{}}{SVAR} models\texorpdfstring{%
      \protect\footnote{Based on \cite[][see pages 332-334]{Hamilton-1994} and \cite[][see chapter 14]{MartinHurnHarris-2012}.}%
  }{}}
Consider a \varp[1]{} model consisting of the logarithm of output and the logarithm of prices:
\[ B_0 y_t = B_1 y_{t-1} + u_t, \quad u_t \sim N(0, D) \]
where
\begin{align*}
    y_t & = \begin{pmatrix} \Delta \ln o_t \\ \Delta \ln p_t \end{pmatrix}
    \\
    B_0 & = \begin{bmatrix}
                1           & 0 \\
                b_{0, 2, 1} & 1
            \end{bmatrix}
    \\
    B_1 & = \begin{bmatrix}
                b_{1, 1, 1} & b_{1, 1, 2} \\
                b_{1, 2, 1} & b_{1, 2, 2}
            \end{bmatrix}
    \\
    D   & = \begin{bmatrix} \sigma_o^2 & 0 \\ 0 & \sigma_p^2 \end{bmatrix}
\end{align*}

\begin{enumerate}
    \item What does the imposition of $b_{0, 1, 2} = 0$ mean?
%
%          \begin{sol}
%              In the short-run, changes in the log prices does not affect the changes in the log output.
%          \end{sol}

    \item Is the \emph{order condition} satisfied?

%          \begin{sol}
%              \begin{definition}
%                  \emph{Order condition:} $B_0$ and $D$ have no more unknown parameters than V.
%              \end{definition}
%
%              \begin{enumerate}[label=$\bullet$]
%                  \item $B_0$ has one unknown parameters
%
%                  \item $D$ has two unknown parameters
%
%                  \item $V$ has three distinct unknown parameters, as it is symmetric matrix
%              \end{enumerate}
%              Therefore, the \emph{order condition} is satisfied.
%          \end{sol}

    \item Determine the following
          \begin{enumerate}[label=\roman*.]
              \item $\theta_B$
              \item $\theta_D$.
              \item $V = \E[v_t v_t'] = (B_0^{-1})' D B_0^{-1}$
              \item $vech(V)$
              \item $J$.
          \end{enumerate}


%          \begin{sol}
%              \begin{flalign*}
%                  \theta_B & = b_{2, 1, 1}
%                  \\[0.8cm] % ======
%                  \theta_D & = \begin{pmatrix}
%                                   \sigma_o^2 \\ \sigma_p^2
%                               \end{pmatrix}
%                  \\[0.8cm] % ======
%                  B_0      & = \begin{bmatrix}
%                                   1           & 0 \\
%                                   b_{0, 2, 1} & 1
%                               \end{bmatrix}
%                  \\[0.8cm] % ======
%                  B_0^{-1} & = \begin{bmatrix}
%                                   1            & 0 \\
%                                   -b_{0, 2, 1} & 1
%                               \end{bmatrix}
%                  \\[0.8cm] % ======
%                  V        & = B_0^{-1} D (B_0^{-1})'
%                  \\[0.8cm] % ==========
%                           & = \begin{bmatrix}
%                                   1            & 0 \\
%                                   -b_{0, 2, 1} & 1
%                               \end{bmatrix}
%                  \begin{bmatrix}
%                      \sigma_o^2 & 0          \\
%                      0          & \sigma_p^2
%                  \end{bmatrix}
%                  \begin{bmatrix}
%                      1 & -b_{0, 2, 1} \\
%                      0 & 1
%                  \end{bmatrix}
%                  \\[0.8cm] % ==========
%                           & = \begin{bmatrix}
%                                   \sigma_o^2
%                                    & 0          \\
%                                   -b_{0, 2, 1} \sigma_o^2
%                                    & \sigma_p^2
%                               \end{bmatrix}
%                  \begin{bmatrix}
%                      1 & -b_{0, 2, 1} \\
%                      0 & 1
%                  \end{bmatrix}
%                  \\[0.8cm]
%                           & = \begin{bmatrix}
%                                   \sigma_o^2
%                                    & - b_{0, 2, 1} \sigma_o^2               \\
%                                   -b_{0, 2, 1} \sigma_o^2
%                                    & -b_{0, 2, 1}^2 \sigma_o^2 + \sigma_p^2
%                               \end{bmatrix}
%                  \\[0.8cm] % ==========
%                  vech(V)  & = \begin{pmatrix}
%                                   \sigma_o^2              \\
%                                   -b_{0, 2, 1} \sigma_o^2 \\
%                                   -b_{0, 2, 1}^2 \sigma_o^2 + \sigma_p^2
%                               \end{pmatrix}
%                  \\[0.8cm] % ==========
%                  J        & = \begin{bmatrix}
%                                   \frac{\partial vech(V)}{\partial \theta_B'}
%                                    & \frac{\partial vech(V)}{\partial \theta_D'}
%                               \end{bmatrix}
%                  \\[0.8cm] % ========
%                           & = \begin{bmatrix}
%                                   0
%                                    & 1
%                                    & 0
%                                   \\
%                                   -\sigma_o^2
%                                    & -b_{0, 2, 1}
%                                    & 0
%                                   \\
%                                   -2 b_{0, 2, 1} \sigma_o^2
%                                    & -b_{0, 2, 1}^2
%                                    & 1
%                                   \\
%                               \end{bmatrix}
%              \end{flalign*}
%          \end{sol}

    \item Using \texttt{R}, check whether the \emph{rank condition} is satisfied at the following parameter values?

          \begin{enumerate}[label=\roman*.]
              \item when $b_{0, 2, 1} = 0.1$, $\sigma_o^2 = 1$, and $\sigma_p^2 = 0.1$

              \item when $b_{0, 2, 1} = 2.1$, $\sigma_o^2 = 0$, and $\sigma_p^2 = 3.1$
          \end{enumerate}
%
%          \begin{sol}
%              \begin{definition}
%                  The rank condition requires that the columns of $J$ are linearly independent.
%              \end{definition}
%              \lstinputlisting{../R-files/rank_of_J_matrix.R}
%          \end{sol}
\end{enumerate}

% Stationarity in univariate case
\section{Stationarity in univariate case\texorpdfstring{%
      \protect\footnote{For more details, see \cite[][see Chapter 3, Stationary ARMA processes]{Hamilton-1994}}%
  }{}}
\begin{enumerate}
    \item Show that an \maq[1]{} process, $y_t = \nu_t + \psi \nu_{t-1}$ with $\nu_t \sim N(0, 1)$, is covariance-stationary.

          \begin{sol}
              We have to check the three conditions from the definition of stationarity, given in the lecture.

              \textbf{Condition 1}, $\E[y_t] = \mu < \infty$:
              \begin{align*}
                  \E[y_t] = \E[\nu_t] + \psi \E[\nu_{t-1}]
                  = 0 + \phi \times 0
                  = 0
              \end{align*}
              Thus, the mean is constant and finite.

              \textbf{Condition 2}, $\Var(y_t) = \sigma^2 < \infty$:
              \begin{align*}
                  \Var(y_t)
                   & = \Var(\nu_t) + \psi^2 \Var(\nu_{t-1})
                  = 1 + \psi^2
              \end{align*}
              Thus, the variance is constant and finite.

              \textbf{Condition 3}, $\Cov(y_t, y_{t-k}) = \gamma_k, k>0$:
              \begin{align*}
                  \Cov(y_t, y_{t-k})
                   & = \Cov(\nu_t + \psi \nu_{t-1}, \nu_{t-k} + \psi \nu_{t-k-1})
                  \\
                   & = \Cov(\nu_t, \nu_{t-k})
                  + \Cov(\nu_t, \nu_{t-k-1})
                  + \Cov(\nu_{t-1}, \nu_{t-k})
                  + \Cov(\nu_{t-1}, \nu_{t-k-1})
                  \\
                   & \Downarrow
                  \\
                  \text{For $k=1$}: \quad \Cov(y_t, y_{t-1})
                   & = 0 + 0 + \Cov(\nu_{t-1}, \nu_{t-1}) + 0 = 1
                  \\
                  \text{For $k=2$}: \quad \Cov(y_t, y_{t-2})
                   & = 0 + 0 + 0 + 0 = 0
                  \\
                  \text{For $k=3$}: \quad \Cov(y_t, y_{t-3})
                   & = 0 + 0 + 0 + 0 = 0
                  \\
                   & \vdots
              \end{align*}
              Thus, the covariance between $y_t$ and $y_{t-k}$ is only a function of the time between two points, and not of time itself.
          \end{sol}

    \item Using R, simulate $2500$ observations from the \arp[1]{} process, $y_t = \phi y_{t-1} + \nu_t$ with $\nu_t \sim N(0, 1)$, for each $\phi = 0.99$, $\phi = 1.00$, and $\phi = 1.01$. Plot the simulated series and the corresponding auto-correlation function (ACF). Try and guess which cases are stationary and which cases are non-stationary. 

    \begin{sol}
        \lstinputlisting{../R-files/simulate_ar1.R}
    \end{sol}

    \item Consider an \armapq{} process, $\Phi(L) X_t = \Theta(L) \nu_t$:
        \begin{enumerate}[label=$\bullet$]
        \item \armapq{} processes are \emph{stable}, if all roots of 
        \[ \Phi(z) = 0 \]
        are larger than 1 in absolute values.

        \item \armapq{} processes are \emph{invertible}, if all roots of 
        \[ \Theta(z) = 0 \]
        are larger than 1 in absolute values.\footnote{For more details, see \cite[][Chapter 3, "Stationary ARMA Processes"]{Hamilton-1994}}
        \end{enumerate}

        Check whether the following \armapq[2, 2]{} is stable and invertible.
        \[
            X_t
            = 0.2 X_{t-1} + 0.48 X_{t-2}
            + \nu_t - 0.2 \nu_{t-1} - 0.08 \nu_{t-2}
        \]

        \begin{sol}
            \begin{align*}
                X_t
                & = 0.2 X_{t-1} + 0.48 X_{t-2}
                + \nu_t - 0.2 \nu_{t-1} - 0.08 \nu_{t-2}
                \\
                X_t - 0.2 X_{t-1} - 0.48 X_{t-2} 
                & = \nu_t - 0.2 \nu_{t-1} - 0.08 \nu_{t-2}
                \\
                (1 - 0.2 L - 0.48 L^2) X_t 
                & = (1 - 0.2 L - 0.08 L^2) \nu_t
            \end{align*}

            Using the following R script, we find that both $\Phi(z) = 0$ and $\Theta(z) = 0$ have roots outside the unit circle.
            \lstinputlisting{../R-files/polynomial_roots.R}
            Thus, the process is both stable and invertible.
        \end{sol}
\end{enumerate}

\section{Stationarity in multivariate case}
\begin{enumerate}
    \item How does the definition of covariance stationarity change in the multivariate case?\footnote{Taken from \cite[][See section 16, "Understanding multivariate time series concepts"]{Mutschler-2018-github_repo}.}

          \begin{sol}
              Basically, it is the same: A stochastic process is called weakly stationary (or covariance stationary), if for each period in time, it has the same expectation and variance, independent of time, and the autocovariance between two points in time is only dependent on the distance of these two points. In the multivariate case, we now also consider the autocovariances between different variables and require these to be only dependent on the distance in time, not on time itself.

              \begin{definition}
                  The K-dimensional stochastic process $y_t$ is called covariance stationary, if for all $h,t,\tau \in \mathbb{Z}$:
                  \begin{align}
                      \E[{y_t}]
                       & = \E[{y_\tau}]
                      \\
                      \Var(y_t) 
                      & = \Var(y_\tau)
                      \\
                      \Cov(y_{t},y_{t-h})
                       & = \Cov(y_{t+\tau}, y_{t-h+\tau})
                  \end{align}
              \end{definition}
          \end{sol}

    \item Interpret $\E[y_t]$ and $\Gamma_h := \Cov(y_{t},y_{t-h})$\footnote{Taken from \cite[][See section 16, "Understanding multivariate time series concepts"]{Mutschler-2018-github_repo}.}.
          \begin{sol}
              \emph{Interpretation of $\E[y_t]$:} Each component of $y_t$ has its own expectation. $\E[y_t]$ is the unconditional expectation of $y_t$ at time point $t$. The expectation as a linear operator can go into a vector or matrix:
              \begin{align*}
                  \E[y_t]
                  = \E\left[
                      \begin{pmatrix}
                          y_{1,t} \cr \vdots \cr y_{K,t}
                      \end{pmatrix}
                      \right]
                  = \begin{pmatrix}
                        \E[y_{1,t}] \cr \vdots \cr \E[y_{K,t}]
                    \end{pmatrix}
              \end{align*}


              \emph{Interpretation of $\Gamma_y(h)$:} In the univariate case, we define the autocovariance as the covariance of a random variable with its own lagged values. In the multivariate case, we also consider covariances between different variables. To this end, we want to summarize the covariance between different variables and different points in time. The Autocovariance matrix \[ {\Gamma_y(h)} := \Cov[{y_{t},{y_{t-h}}}] \] sums this information up in a neat fashion and is therefore a powerful tool in multivariate time series analysis:
              \begin{align*}
                  \Cov[{y_{t}, {y_{t-h}}}]
                   & = \E\left[
                      ({y_{t}} - \E[{{y_{t}}}])
                      ({y_{t-h}} - \E[{{y_{t-h}}}])'
                      \right]
                  \\[0.5cm]
                   & = \E\left[
                      \begin{pmatrix}
                          y_{1,t} - \E[y_{1,t}]
                          \cr \vdots
                          \cr y_{K,t} - \E[y_{K,t}]
                      \end{pmatrix}
                      \begin{pmatrix}
                          y_{1,{t-h}} - \E[y_{1,{t-h}}]
                           & \dots
                           & y_{K,{t-h}} - \E[y_{K,{t-h}}]
                      \end{pmatrix}
                      \right]
                  \\[0.5cm]
                   & = \E\left[
                      \begin{pmatrix}
                          (y_{1,t}- \E[y_{1,t}])(y_{1,{t-h}} - \E[y_{1,{t-h}}])
                           & \dots
                           & (y_{1,t} - \E[y_{1,t}])(y_{K,{t-h}} - \E[y_{K,{t-h}}])
                          \\
                          \vdots
                           & \ddots
                           & \vdots
                          \\
                          (y_{K,t}- \E[y_{K,t}])(y_{1,{t-h}} - \E[y_{1,{t-h}}])
                           & \dots
                           & (y_{K,t} - \E[y_{K,t}])(y_{K,{t-h}} - \E[y_{K,{t-h}}])
                      \end{pmatrix}
                      \right]
                  \\[0.5cm]
                   & = \begin{pmatrix}
                           \E[(y_{1,t}- \E[y_{1,t}])(y_{1,{t-h}} - \E[y_{1,{t-h}}])]
                            & \dots
                            & \E[(y_{1,t} - \E[y_{1,t}])(y_{K,{t-h}} - \E[y_{K,{t-h}}])]
                           \\
                           \vdots
                            & \ddots
                            & \vdots
                           \\
                           \E[(y_{K,t}- \E[y_{K,t}])(y_{1,{t-h}} - \E[y_{1,{t-h}}])]
                            & \dots
                            & \E[(y_{K,t} - \E[y_{K,t}])(y_{K,{t-h}} - \E[y_{K,{t-h}}])]
                       \end{pmatrix}
                  \\[0.5cm]
                   & =\begin{pmatrix}
                          \Cov(y_{1,t},y_{1,{t-h}})
                           & \dots
                           & \Cov(y_{1,t},y_{K,{t-h}})
                          \\
                          \vdots
                           & \ddots
                           & \vdots
                          \\
                          \Cov(y_{K,t},y_{1,{t-h}})
                           & \dots
                           & \Cov(y_{K,t},y_{K,{t-h}})
                      \end{pmatrix}
              \end{align*}
              On the diagonals, we have the autocovariance of each variable, on the off-diagonals we have the covariances between the different variables.
          \end{sol}

    \item Show that the following \varp[2]{} process is covariance stationary.

    \begin{sol}
        
    \end{sol}
\end{enumerate}


% Theoretical and simulated moments
\section{Theoretical and simulated moments\texorpdfstring{%
      \protect\footnote{Taken from \cite[][see Section 18, ``Theoretical and simulated moments"]{Mutschler-2018-github_repo}}%
  }{}}
\begin{enumerate}
\item Derive the theoretical mean ($\mu_y$), covariance matrix ($\Gamma(0)$) and autocovariance matrix ($\Gamma(h)$) of the covariance stationary \varp[1]{} model: $y_t = \mu + \Phi y_{t-1} + \nu_t, \quad \nu_t \sim N(0, V)$

\begin{sol}
    \textbf{Mean:} \\
    \begin{align*}
        \underbrace{\E(y_t)}_{\mu_y}
        & = \underbrace{\E(\mu)}_{\mu}
        + \underbrace{\E(\Phi y_{t-1})}_{\Phi \mu_y}
        + \underbrace{\E(\nu_t)}_{0}
        \\
        \Leftrightarrow \mu_y      
        & = (I - \Phi)^{-1} \mu
    \end{align*}
    Put $\mu = (I - \Phi) \mu_y:
    \underbrace{y_t -\mu_y}_{ \tilde{y}_{t}}
    = \Phi (\underbrace{y_{t-1} - \mu_y}_{\tilde{y}_{t-1}})
    + u_t$.


    \textbf{Covariance matrix and Autocovariance matrix:} \\
    Note: $\Gamma_y(h) = \Gamma_{\tilde{y}}(h)$
    \begin{align*}
        \Gamma_y(0)
        & = \E\left[\tilde{y}_t \tilde{y}_t'\right]
        = \E\left[(\Phi \tilde{y}_{t-1} + \nu_t)(\Phi \tilde{y}_{t-1} + \nu_t)'\right]
        \\
        & = \Phi \underbrace{\E\left[\tilde{y}_{t-1}\right]}_{\Gamma_y(0)} \Phi'
        + \Phi \underbrace{\E\left[\tilde{y}_{t-1}\nu_t'\right]}_{0}
        + \underbrace{\E\left[\nu_t \tilde{y}_{t-1}'\right]}_{0} \Phi'
        + \underbrace{\E\left[\nu_t \nu_t'\right]}_{V}
        \\
        \Leftrightarrow \Gamma_y(0)
        & = \Phi \Gamma_y(0) \Phi' + V
    \end{align*}
    This is a discrete Lyapunov matrix equation, see exercise 1.5:
    \[
        vec\bigg( \Gamma_y(0) \bigg)
        = \bigg( I - \Phi \otimes \Phi \bigg)^{-1} vec\bigg( \Sigma_u \bigg)
    \]
    \begin{align*}
        \Gamma_y(1)
        & = \E\left[\tilde{y}_t \tilde{y}_{t-1}'\right]
        = \E\left[(\Phi \tilde{y}_{t-1} + \nu_t)\tilde{y}_{t-1}'\right]
        = \Phi \E\underbrace{
            \left[\tilde{y}_{t-1}\tilde{y}_{t-1}'\right]
        }_{\Gamma_y(0)}
        + \underbrace{\E\left[\nu_t y_{t-1}'\right]}_{0}
        = \Phi \Gamma_y(0)
        \\
        \Gamma_y(2)
        & =  \E\left[\tilde{y}_t \tilde{y}_{t-2}'\right]
        = \E\left[(\Phi \tilde{y}_{t-1} + \nu_t)\tilde{y}_{t-2}'\right]
        = \Phi \Gamma_y(1) = \Phi^2 \Gamma_y(0)
        \\
        \Gamma_y(h)
        & = \Phi^h \Gamma_y(0)
    \end{align*}
\end{sol}

\item Simulate $R = 100$ datasets each with $T = 100$ observations for
\begin{align*}
    y_t = \begin{pmatrix}0.2 &0.3 \\-0.6 & 1.1 \end{pmatrix} y_{t-1}  + \nu_t
\end{align*}
provided that $\nu_t \sim N(0, V)$ and $V = \begin{pmatrix}
    0.9 & 0.2 \\ 0.2 & 0.5
\end{pmatrix}$.

Compute the sample mean and sample covariance matrix. Compare the results with parts (1) and (2). How does your choice of $R$ or $T$ change results?

\begin{sol}
  \lstinputlisting{../R-files/var1_simulate_moments.R}
    The Monte Carlo shows that either increasing $R$ or $T$ (or both) captures the moments better.
\end{sol}
\end{enumerate}

%% Transforming var to vma
%\section{Transforming a VAR to a VMA}
%\begin{enumerate}
    \item Using R, plot the Auto-correlation function (ACF) and the Partial Auto-correlation function (PACF) of the follwoing \ar{} and \ma{} models.

    \begin{sol}
        
    \end{sol}

    \item Finding the coefficients in the univariate case

    \begin{sol}

    \end{sol}

    \item Consider the following VAR(1) process $y_t = \mu + \Phi_1 y_{t-1} + \nu_t$
          \begin{align*}
              \begin{pmatrix} y_{1,t} \\ y_{2,t} \\ y_{3,t} \end{pmatrix}
               & = \begin{pmatrix} 0 \\ 0 \\ 0 \end{pmatrix}
              + \begin{pmatrix}
                    0.5 & 0   & 0   \\
                    0.1 & 0.1 & 0.3 \\
                    0   & 0.2 & 0.3
                \end{pmatrix}
              \begin{pmatrix}
                  y_{1, {t-1}} \\ y_{2, {t-1}} \\ y_{3, {t-1}}
              \end{pmatrix}
              + \begin{pmatrix}
                    \nu_{1,t} \\ \nu_{2,t} \\ \nu_{3,t}
                \end{pmatrix}
          \end{align*}
          provided that $\nu_t \sim N(0, V)$ and
          \[
              V
              = \begin{pmatrix}
                  2.25 & 0 & 0 \\ 0 & 1 & 0.5 \\ 0 & 0.5 & 0.74
              \end{pmatrix}
          \]
          Compute the coefficients $\Psi_0, \Psi_1, \dots \in \mathbb{R}^{3\times 3}$ of the lag polynomial $\Psi(L) := \sum_{i=0}^\infty \Psi_i L^i$, and a $c \in \mathbb{R}^3$ such that\footnote{Taken from \cite[][See section 16, "Understanding multivariate time series concepts"]{Mutschler-2018-github_repo}.}
          \begin{align*}
              y_t = c + \Psi(L) \nu_t
          \end{align*}

          \begin{sol}
              We will use the very efficient method of matching coefficients to transform the \varp[1]{} model into a \vmaq[$\infty$]{} representation.
              \begin{align*}
                  y_t
                   & = \Phi_1 y_{t-1} + \nu_t
                  \\
                  \underbrace{(I_3 - \Phi_1 L)}_{\Phi(L)} y_t
                   & = \nu_t
                  \\
                  \Phi(L) y_t
                   & = \nu_t
                  \\
                  \Phi(L)^{-1} \Phi(L) y_t
                   & = \Phi(L)^{-1} \nu_t
                  \\
                  y_t
                   & = \Phi(L)^{-1} \nu_t
                   = c + \Psi(L) \nu_t
                  \\
                  \Rightarrow
                   & c = 0,
                   \Psi(L) = \Phi(L)^{-1}
              \end{align*}
              In the method of matching coefficient we compare the coefficient matrices multiplied to each power of $L$. That is, the expression on the left hand side has to match the expression on the right hand side. In our case:
              \begin{align*}
                  \Psi(L)
                   & = \Phi(L)^{-1}
                  \\
                  \Phi(L) \Psi(L)
                   & = I_3
                  \\
                  (I_3 - \Phi_1 L) \left(\sum_{i=0}^\infty \Psi_i L^i\right)
                   & = I_3
              \end{align*}

              Expanding the two brackets:
              \begin{align*}
                  \Psi_0 L^0
                   & + \Psi_1 L^1 + \Psi_2 L^2 + \dots
                  \\
                   & - \Phi_1 \Psi_0 L^1 - \Phi_1 \Psi_1 L^2 - \dots = I_3 L^0
              \end{align*}

              Finally, let us compare the expressions on the left hand side to those on the right hand side:
              \begin{align}
                  L^0
                   & : \Phi_0 = I_3
                   \Rightarrow
                   \Phi_0 = I_3
                  = \begin{pmatrix}
                        1 & 0 & 0 \\ 0 & 1 & 0 \\ 0 & 0 & 1
                    \end{pmatrix}
                    \nonumber \\
                  L^1
                   & : \Phi_1 - A \Phi_0 = 0
                   \Rightarrow
                   \Phi_1 = A \Phi_0
                  = A
                  = \begin{pmatrix}
                        0.5 & 0 & 0 \\ 0.1 & 0.1 & 0.3 \\ 0 & 0.2 & 0.3
                    \end{pmatrix}
                  \nonumber \\
                  L^2
                   & : \Phi_2 - A \Phi_1 = 0 
                   \Rightarrow
                   \Phi_2 = A \Phi_1
                  = A^2
                  = \begin{pmatrix}
                        0.25 & 0 & 0 \\ 0.06 & 0.07 & 0.12 \\ 0.02 & 0.08 & 0.15
                    \end{pmatrix}
                  \nonumber \\
                  & \vdots
                  \nonumber \\
                  \text{In general} & : \Phi_s = A^s \label{allg}
              \end{align}
          \end{sol}
\end{enumerate}
%
%% Monte Carlo estimation of var models
%\section{Monte Carlo estimation of VAR models\texorpdfstring{%
%      \protect\footnote{Based on \cite{Mutschler-2018-github_repo} and \cite{Hamilton-1994}}%
%  }{}}
%Maximum Likelihood estimation of \varp{} model is equivalent to OLS estimation under the assumption of normality for the error term. In the case of \varp[1]{} model, estimate of the transition matrix $\Phi_1$ is then:
\[
    \hat\Phi_1 = (X'X)^{-1} X'y
\]
where
\begin{align*}
    y & = \begin{bmatrix}
              y_{1, 1}   & y_{2, 1}   & \ldots & y_{N, 1}   \\
              y_{1, 2}   & y_{2, 2}   & \ldots & y_{N, 2}   \\
                         &            & \vdots &            \\
              y_{1, T-1} & y_{2, T-1} & \ldots & y_{N, T-1}
          \end{bmatrix}
    \\[0.8cm]
    X & = \begin{bmatrix}
              1 & y_{1, 2} & y_{2, 2} & \ldots & y_{N, 2} \\
              1 & y_{1, 3} & y_{2, 3} & \ldots & y_{N, 3} \\
                &          & \vdots   &                   \\
              1 & y_{1, T} & y_{2, T} & \ldots & y_{N, T}
          \end{bmatrix}.
\end{align*}

The covariance matrix of the error term is estimated using residuals, $\hat \nu_t$:
\[
    \hat V = \frac{1}{T-1} U' U
\]
with
\begin{align*}
    U & = \begin{bmatrix}
              \hat\nu_{1, 2} & \hat\nu_{2, 2} & \ldots & \hat\nu_{N, 2} \\
              \hat\nu_{1, 3} & \hat\nu_{2, 3} & \ldots & \hat\nu_{N, 3} \\
                             &                & \vdots &                \\
              \hat\nu_{1, T} & \hat\nu_{2, T} & \ldots & \hat\nu_{N, T}
          \end{bmatrix}
\end{align*}

\begin{enumerate}
    \item Simulate a dataset with $T=250$ using the following model, and estimate it with the approach described above:
          \begin{align*}
              \mu    & = \begin{pmatrix} 0 \\ 0 \end{pmatrix}
              \\
              \Phi_1 & = \begin{bmatrix}
                             0.3 & 0.1 \\
                             0.2 & 0.4
                         \end{bmatrix}
              \\
              V      & = \begin{bmatrix}
                             0.23  & -0.10 \\
                             -0.10 & 0.32
                         \end{bmatrix}
          \end{align*}

          \begin{sol}
              \lstinputlisting{../R-files/monte_carlo_estimation_var1.R}
          \end{sol}

    \item Simulate $R = 100$ datasets, each with $T = 250$ observations with the model given in the subtask above. Estimate the elements of $\Phi_1$ with each data set and plot the histogram of the estimates. Do the same for $T = 2500, 25000$. Using the histogram, explain the consistency and asymptotic normality of MLE/OLS.

          \begin{sol}
              \lstinputlisting{../R-files/monte_carlo_estimation_var1.R}

              As can be seen from the results, as we have more data points our histograms become narrower horizontally. This means the variance decreases and thus shows the consistency.

              The histograms are bell-shaped curves around the true values. The bell-shape gets better and better as $T$ increases. This is asymptotic normality.
          \end{sol}
\end{enumerate}
%
% Estimation of Okun's Law
\section{Estimation of Okun's law\texorpdfstring{%
      \protect\footnote{Based on \cite{KissNguyenOesterholm-2023}}%
  }{}}
\label{ex-Okun_law-var}
Okun's Law describes the macroeconomic relationship between GDP and UNEMPLOYMENT. In this exercise, you will estime this law using \verb|vars| package. Use the following series for the estimation:
\begin{enumerate}[label=-]
    \item \verb|CLVMNACSCAB1GQDE.csv|: Real Gross Domestic Product for Germany; from FRED; seasonally adjusted; Quarterly; Millions of Chained 2010 Euros

    \item \verb|LRHUADTTDEQ156S.csv|: Unemployment rate for Germany; from FRED; seasonally adjusted; Quarterly; percent
\end{enumerate}

\begin{enumerate}
    \item Read both data series from the \texttt{.csv} files.

          \begin{sol}
              See \texttt{okuns\_law.R}. You will find this file on the Learnweb and also at the last solution.
          \end{sol}

    \item Transform the \emph{GDP} into \emph{growth rate} using the formula below:
          \begin{align}
              g_t = 100 (Y_t / Y_{t-1} - 1) \label{eq-gdp-growth_rate}
          \end{align}
          where $g_t$ is growth rate, $Y_t$ is seasonally adjusted real GDP

          \begin{sol}
              See \texttt{okuns\_law.R}. You will find this file on the Learnweb and also at the last solution.
          \end{sol}

    \item Transform the \emph{unemployment rate} into the \emph{change in unemployment rate} by taking the first difference.

          \begin{sol}
              See \texttt{okuns\_law.R}. You will find this file on the Learnweb and also at the last solution.
          \end{sol}

    \item Plot the both series with and without the transformation, in order to have the first impression. Do the transformed data series look stationary? What would you do to check the stationarity formally?

          \begin{sol}
              First, see \texttt{okuns\_law.R}. You will find this file on the Learnweb and also at the last solution.

              The data series that are not transformed look non-stationary, while tranformed ones look stationary. You can formally check for stationarity formally, using tests such as Augmented Dickey-Fuller (ADF) or Kwiatkowski-Phillips-Schmidt-Shin (KPSS). Usually, these two tests are conducted together.
          \end{sol}

    \item Select the optimal number of lags using information criteria. (Use \verb|VARselect()| from \verb|vars| package.)

          \begin{sol}
              See \texttt{okuns\_law.R}. You will find this file on the Learnweb and also at the last solution.

              The R-codes provide the following output:
              \input{../Tex-files-auto/okun_law-lag_sel_table.tex}

              Based on Information criteria, we should select only one lag.
          \end{sol}

    \item Estimate \varp{} model for the Okun's law using the command \verb|VAR()| from \verb|vars| package. Use the lag order you chose above.

          \begin{sol}
              See \texttt{okuns\_law.R}. You will find this file on the Learnweb and also at the last solution.

              \verb|vars| package estimates the \varp{} models, equation by equation. Thus, the code above results in the following two tables:

              The regression equation for unemployment is:
              % latex table generated in R 4.4.0 by xtable 1.8-4 package
% Mon Oct 21 16:16:10 2024
\begin{table}[ht]
\centering
\begin{tabular}{rrrrr}
  \hline
 & Estimate & Std. Error & t value & Pr($>$$|$t$|$) \\ 
  \hline
unemployment.l1 & 0.8102 & 0.0487 & 16.64 & 0.0000 \\ 
  growth.l1 & -0.0166 & 0.0069 & -2.40 & 0.0178 \\ 
  const & 0.0019 & 0.0096 & 0.20 & 0.8439 \\ 
   \hline
\end{tabular}
\end{table}


              The regression equation for gdp growth is:
              % latex table generated in R 4.4.0 by xtable 1.8-4 package
% Mon Oct 21 16:16:10 2024
\begin{table}[ht]
\centering
\begin{tabular}{rrrrr}
  \hline
 & Estimate & Std. Error & t value & Pr($>$$|$t$|$) \\ 
  \hline
unemployment.l1 & -0.4685 & 0.6219 & -0.75 & 0.4526 \\ 
  growth.l1 & -0.1632 & 0.0882 & -1.85 & 0.0664 \\ 
  const & 0.3527 & 0.1221 & 2.89 & 0.0045 \\ 
   \hline
\end{tabular}
\end{table}

          \end{sol}

    \item Interpret your results. Do the results seem appropriate?

          \begin{sol}
              Decreasing the unemployment increases the growth. So, the sign makes sense. However, unemployment is not statistically significant.

              It might be that there is a contemporaneous relationship between unemployment and GDP. However, to analyze this, we need to estimate a \svarp{} model. We will estimate such a model in the next exercise.
          \end{sol}

    \item Check if the \emph{change in unemployment rate} Granger-cause the \emph{GDP growth}?

          \begin{sol}
              See \texttt{okuns\_law.R}. You will find this file on the Learnweb and also at the last solution.

              The R-codes provide the following result: \texttt{\input{../Tex-files-auto/granger_res.tex}}

              We cannot reject the $H_0$ and therefore, we cannot find any evidence for the Granger-causality.

              Note that the $p$-value we have above is very similar to the $p$-value associated with \texttt{unemployment.l1} in the regression summary for growth. This is because, there is only one lag and $t$-test almost coincides with testing the Granger-causality.
          \end{sol}

    \item Using \verb|irf(.)| function from \verb|vars|, evaluate the first ten impulse response coefficients. (Set the option \verb|ortho| to \texttt{FALSE}. We will set it to \texttt{TRUE} later on, for \svarp{} models.)

          Plot evaluated IRF and the associated confidence intervals as line.

          \begin{sol}
              See \texttt{okuns\_law.R}. You will find this file on the Learnweb and also at the last solution.
          \end{sol}

    \item Find the impulse-response function (or coefficients) "from unemployment" to "growth" by writing the codes yourselves.

          \begin{sol}
              The R-codes are:
              %   \lstinputlisting{../R-files/okuns_law.R}
          \end{sol}
\end{enumerate}


% SVAR: recursive identification
\section{Recursive Identification of SVAR models\texorpdfstring{%
      \protect\footnote{Based on \cite[][see Section 29, ``Recursively Identified Models By Short-Run Restrictions"]{Mutschler-2018-github_repo}}%
  }{}}
In this exercise, we will build and estimate a \svarp{} model to analyze Okun's law. For identification, we need to impose restrictions and in this exercise, we want to impose recursive restrictions.

\begin{enumerate}
    \item We want to impose recursive restrictions such that (i) the \emph{unemployment rate} is predetermined and (ii) the \emph{unemployment rate} affects the \emph{growth rate}.
          \[ B_0 y_t = B_1 y_{t-1} + u_t \]

          \begin{enumerate}[label=\roman*.]
              \item Why imposing zeros on elements of $B_0$ a short-run restriction?

                    \begin{sol}
                        $B_0$ captures the contemporaneous influence of the elements of $y_t$ on each other. Therefore, it is a short-run restriction.
                    \end{sol}

              \item Determine $y_t$ and $B_0$.

                    \begin{sol}
                        The \emph{unemployment rate} should come first, as it is predetermined.

                        The \emph{growth rate} is affected only by the \emph{unemployment rate}. So, it should come second.

                        As a result,
                        \begin{align*}
                            y_t & = \begin{pmatrix} l_t \\ g_t \end{pmatrix}
                            \\[0.2cm]
                            B_0 & = \begin{bmatrix}
                                        1           & 0 \\
                                        b_{0, 2, 1} & 1
                                    \end{bmatrix}
                        \end{align*}
                        where, $g_t$ is growth rate as defined in the equation \ref{eq-gdp-growth_rate} and $l_t$ is percentage change in the unemployment rate.
                    \end{sol}

              \item Can you interpret all the model shocks with recursive identification?

                    \begin{sol}
                        With recursive identification procedure, we focus on the effect of an unanticipated increase in the percentage change of \emph{unemployment rate} on the \emph{growth rate}. Only the shock of the \emph{unemployment rate} can be given an economic interpretation.
                    \end{sol}
          \end{enumerate}

    \item Estimate the reduced-form vector autoregressive model with \verb|vars| package. Estimate the structural impact multiplier matrix $B_0^{-1}$ based on a lower-triangular Cholesky decomposition of the residual covariance matrix.

          \begin{sol}
              The matrix $V$ of the reduced form can be estimated as before, using the methods from the reduced form estimation.\footnote{Taken from \cite[][see page 495]{MartinHurnHarris-2012}}. The estimation result is then $\hat V$.

              The Cholesky decomposition of $V$ gives us the lower triangular matrix $S$ and its transpose:
              \[ V = S S' \Rightarrow \hat V = \hat S \hat S' \]

              The matrix $D$ is a diagonal matrix. Diagonal elements of $D$ is the diagonal elements of $S$. That is how we obtain $\hat D$.

              Using the matrices $\hat S$ and $\hat D$, we can find $\hat B_0$ as follows:
              \begin{align*}
                  S   & = B_0^{-1} D^{1/2}  \\
                      & \Downarrow          \\
                  B_0 & = (S D^{-1/2})^{-1}
              \end{align*}

              \lstinputlisting{../R-files/recur_iden.R}
          \end{sol}

    \item Estimate the structural vector autoregressive model using \verb|SVAR()| command from \verb|vars| package. Use direct minimization of the likelihood and choose \verb|BFGS| as your numerical minimization algorithm.

          \begin{sol}
              The solution is in the R-file given in the solution above.
          \end{sol}

    \item Plot the impulse response function of an unexpected increase in the percentage change in the unemplyoment rate, using \verb|vars| package.

          Interpret IRFs economically.

          Then, evaluate the same IRFs manually. Compare your results with those from \verb|vars|.

          \begin{sol}
              For the usage of \verb|vars|, see the R-codes. Here, we will explain how to obtain orthogonalized IRFs manually.
              \begin{align*}
                  B_0 Y_t
                   & = \mu + B_1 y_{t-1} + \ldots + u_t
                  \\
                  Y_t
                   & = B_0^{-1} \mu
                  + \underbrace{B_0^{-1} B_1}_{\Phi_1} y_{t-1}
                  + \ldots
                  + B_0^{-1} D \eta_t
              \end{align*}
              where $\eta_t \sim N(0, I)$.

              Orthogonalized impulse-response means having an impulse in $\eta_t$ rather than in the reduced form shocks.

              You already learned the theoretical aspects and definitions of IRFs in the lecture. Now, we concentrate on obtaining IRFs practically: (i) set first element of $\eta_t$ to $1$ (which corresponds to percentage change in oil price), keeping other elements at zero; (ii) Set $y_{t-1}$, $y_{t-2}$, $\ldots$ to zero; (iii) Run the system without $B_0^{-1} \mu$ for $h$ steps ahead. More details are availabe at \cite[][see pages 319, 322-323]{Hamilton-1994}.

              Interpretation: One-standard deviation shock in the percentage change of unemployment has the effects depicted in the plots.
          \end{sol}

    \item Plot the impulse response function of an unexpected increase in the unemployment rate (not its percentage change!) using \verb|vars| package.

          \begin{sol}
              Set \verb|cumulative = TRUE| in the \verb|irf()| command.

              This is equivalent to using \verb|cumsum()| on your manually computed IRFs.
          \end{sol}

    \item Evaluate the Variance Decomposition for $h=3$, first manually, using the estimation results you obtained above. Then, evaluate it using \verb|vars| package. Compare both results. Lastly, interpret the variance decompositions you obtained.

          \begin{sol}
              To obtain variance decomposition we use the formula from \cite[][see page 499]{MartinHurnHarris-2012}:
              \begin{align*}
                  VD_h = \sum_i^h IRF_i \odot IRF_i
              \end{align*}
              See the R-codes for its implementation.

              The first row tells us that the entire variance of $h = 1$ step ahead forecast error in the output equation of the \varp{} is the contribution of unemployment shock. The shock in GDP does not contribute at all

              The second row tells that $0.9775\%$ of the forecast error variance at $h = 2$ is explained by the shock in unemployment. Only $0.0225\%$ is due to the shock in GDP.

              The third row can be interpreted in this fashion as well.

              For more information about interpreting Variance Decomposition, see page 500 of \cite{MartinHurnHarris-2012}.
          \end{sol}
\end{enumerate}

% SVAR: short-run restrictions, long-run restrictions, and their combination
\section{Restrictions in SVAR\texorpdfstring{ %
      \protect\footnote{Based on \cite{MartinHurnHarris-2012,Mutschler-2018-github_repo}} %
  }{}}
In this exercise, we will analyze output and prices in Germany using structural \var{} models. We will first use short run restriction and then long run restriction. Finally, we will combine them.

The model is:
\[ B_0 y_t = B_1 y_{t-1} + u_t \]
with
\[ y_t = \begin{pmatrix} \Delta ln o_t \\ \Delta ln p_t \end{pmatrix}. \]

Use the following data:
\begin{enumerate}[label=-]
    \item \verb|CLVMNACSCAB1GQDE.csv|: Real Gross Domestic Product for Germany; from FRED; seasonally adjusted; Quarterly; Millions of Chained 2010 Euros

    \item \verb|CP0000DEM086NEST|: Harmonized Index of Consumer Prices: All Items for Germany; from FRED; Index $2015=100$; Not Seasonally Adjusted; Quarterly - End of Period; Adjusted
\end{enumerate}



\begin{enumerate}
    \item Read, plot, and transform the data.

          \begin{sol}
              See the R-codes
          \end{sol}

    \item Short-run restrictions can be imposed either on $B_0$ or $S = B_0^{-1} D^{1/2}$. What do restrictions on either matrix mean?

          \begin{sol}
              $B_0$ captures the contemporaneous influence of the elements of $y_t$ on each other.

              $S$ captures the contemporaneous relationships between the $y_t$ variables and the structural shocks.\footnote{\cite[see page 518][]{MartinHurnHarris-2012}}.

              Note that, in the recursive case restrictions on $S$ and $B_0$ coincide.
          \end{sol}

    \item We want to impose that in the short run, real shocks do not affect the nominal variables. Determine the matrix $S$ and estimate the model.

          \begin{sol}
              \[
                  S = \begin{bmatrix}
                      s_{1, 1} & s_{1, 2} \\
                      0        & s_{2, 2}
                  \end{bmatrix}
              \]

              The rest is implemented in the R-codes.
          \end{sol}

    \item We want to impose that in the long run, nominal shocks do not affect the real variables. Determine the corresponding matrices of the \svarp{} model and estimate.

          \begin{sol}
              \[
                  F = \Phi(1)^{-1} S =  \begin{pmatrix}
                      f_{1, 1} & 0        \\
                      f_{2, 1} & f_{2, 2}
                  \end{pmatrix}
              \]
          \end{sol}

    \item Now we want to combine the short run and long run restrictions we had above. Determine the corresponding matrices and estimate.

          \begin{sol}
              \begin{align*}
                  S
                   & = \begin{bmatrix}
                           s_{1, 1} & s_{1, 2} \\
                           0        & s_{2, 2}
                       \end{bmatrix}
                  \\[0.2cm]
                  F
                   & =  \begin{pmatrix}
                            f_{1, 1} & 0        \\
                            f_{2, 1} & f_{2, 2}
                        \end{pmatrix}
                  \\[0.2cm]
                  \Phi(1)^{-1}
                   & = \begin{pmatrix}
                           \phi^{1, 1} & \phi^{1, 2} \\
                           \phi^{2, 1} & \phi^{2, 2}
                       \end{pmatrix}
                  \\[0.2cm]
                  F
                   & = \Phi(1)^{-1} S
                  \\[0.2cm]
                   & \Downarrow
                  \\[0.2cm]
                  0
                   & = f_{1, 2} = \phi^{1, 1} s_{1, 2} + \phi^{1, 2} s_{2, 2}
                  \\[0.2cm]
                  s_{2, 2}
                   & = -\frac{\phi^{1, 1}}{\phi^{1, 2}} s_{1, 2}
                  \\[0.2cm]
                   & \Downarrow
                  \\[0.2cm]
                  S
                   & = \begin{pmatrix}
                           s_{1, 1} & s_{1, 2}
                           \\
                           0        & -\frac{\phi^{1, 1}}{\phi^{1, 2}} s_{1, 2}
                       \end{pmatrix}
              \end{align*}

              As the model is over-identified, the non-linear equation solver from above cannot be used anymore. Instead, we need to maximize the following concentrated log-likelihood:
              \[
                  \ln L(\theta) = -\frac{N}{2} \ln 2 \pi
                  - \frac{1}{2} \ln |V|
                  - \frac{1}{2 (T - p)} \sum_{t = p + 1}^T \hat \nu_t' V^{-1} \hat \nu_t
              \]
              $\hat \nu_t$ comes from the OLS estimation which is the first stage of estimation. Moreover, in the actual estimation, $\bigg(-\frac{N}{2} \ln 2 \pi \bigg)$ can be left out, as it is just a constant number and does not affect the maximization.

              The estimation is in the following R-file:
              \lstinputlisting{../R-files/short_long_iden.R}
          \end{sol}
\end{enumerate}


% Estimating Peersman's Model of Oil Price Shocks
\section{Estimating Peersman's Model of Oil Price Shocks\texorpdfstring{ %
      \protect\footnote{Based on \cite{MartinHurnHarris-2012,Mutschler-2018-github_repo}} %
  }{}}
In this exercise, we estimate the Peersman's model presented in the lecture.\footnote{For textbook treatment, \cite[][see page 535]{MartinHurnHarris-2012}.}

The model is:
\[ B_0 y_t = B_1 y_{t-1} + u_t \]
with
\[
    y_t = \begin{pmatrix}
        \Delta ln oil_t \\ \Delta ln o_t, \\ \Delta ln p_t \\ r_t
    \end{pmatrix}.
\]

You have already seen in the lecture, the $S$-matrix is as follows, after using the short and long run restrictions combined:
\[
    \begin{bmatrix}
        s_{1, 1}
         & 0
         & 0
         & 0
        \\
        s_{2, 1}
         & s_{2, 2}
         & s_{2, 3}
         & 0
        \\
        s_{3, 1}
         & s_{3, 2}
         & s_{3, 3}
         & -\frac{\phi^{2, 4}}{\phi^{2, 3}} s_{4, 4}
        \\
        s_{4, 1}
         & s_{4, 2}
         & -\frac{\phi^{2, 2}}{\phi^{2, 4}} s_{2, 3}
        -\frac{\phi^{2, 3}}{\phi^{2, 4}} s_{3, 3}
         & s_{4, 4}
    \end{bmatrix}
\]

Use the following data:
\begin{enumerate}[label=--]
    \item \verb|DCOILWTICO|: Crude Oil Prices: West Texas Intermediate (WTI) - Cushing, Oklahoma; from FRED; Not Seasonally Adjusted; Quarterly - Average; Index Jan $2010=100$

    \item \verb|CLVMNACSCAB1GQDE.csv|: Real Gross Domestic Product for Germany; from FRED; seasonally adjusted; Quarterly; Millions of Chained 2010 Euros

    \item \verb|CP0000DEM086NEST|: Harmonized Index of Consumer Prices: All Items for Germany; from FRED; Index $2015=100$; Not Seasonally Adjusted; Quarterly - End of Period; Adjusted

    \item \verb|IR3TIB01DEM156N|: Interest Rates: 3-Month or 90-Day Rates and Yields: Interbank Rates: Total for Germany; from FRED; Percent; Not Seasonally Adjusted; Quarterly - Average
\end{enumerate}


\begin{enumerate}
    \item Read, transform, and plot the data.

          \begin{sol}
              See the R-codes.
          \end{sol}

    \item Estimate the model.

          \begin{sol}
              \lstinputlisting{../R-files/peersman_model.R}
          \end{sol}
\end{enumerate}


% Unit root testing with ADF test
\section{Unit root testing with ADF test}
In our previous exercises, we plotted the data series and qualitatively judged whether they look stationary or not. In this exercise, we will check them formally using ADF test.

Following \cite[][see chapter 17]{Hamilton-1994}, ADF-test can be summarized as follows:

\begin{tabular}{r|llll}
     & Estimated process
     & True process
     & $H_0$
     & Test stat.
    \\
    \toprule
    Case 1:
     & $y_t = \rho y_{t-1} + u_t$
     & $y_t = y_{t-1} + u_t$
     & $\rho = 1$
     & $\tau_1$
    \\
    Case 2:
     & $y_t = \alpha + \rho y_{t-1} + u_t$
     & $y_t = y_{t-1} + u_t$
     & $\rho = 1$ (and $\alpha = 0$)
     & $\tau_2$ (and $\Phi_1$)
    \\
    Case 3:
     & $y_t = \alpha + \rho y_{t-1} + u_t$
     & $y_t = \alpha + y_{t-1} + u_t$; $\alpha \neq 0$
     & $\rho = 1$
     &
    \\
    Case 4:
     & $y_t = \alpha + \rho y_{t-1} + \delta t +  u_t$
     & $y_t = \alpha + y_{t-1} + u_t$; $\alpha$ any
     & $\rho = 1$ (and $\delta = 0$)
     & $\tau_3$ (and $\Phi_3$)
\end{tabular}

The asymptotic properties of the \ols{} estimate $\hat\rho$ when the true value of $\rho$ is unity, depend on which case we want to use.

Which is the "correct case" to use? The answer depends on why we are interested in testing for a unit root:
\begin{enumerate}[label=(\arabic*)]
    \item If the analyst has a specific null Hypothesis about the data generating process, then this would guide the choice of the test.

    \item If there is no such a guidance, a general principle would be to fit a specification that is plausible description of the data under null and alternative Hypotheses.
\end{enumerate}

For analyzing unit roots, we will use an r package called \texttt{urca}, from \cite{Pfaff-2008}. The names in the "Test stat." column of the table above, are used in the outputs generated by the commands of this package. $\tau_1$, $\tau_2$, $\tau_3$ refer to tests of only $\rho = 1$ in each respective case. The joint test are called $\Phi_1$, $\Phi_2$, and $\Phi_3$. $\Phi_2$ is the joint test of $\alpha = 0$, $\rho = 1$, and $\delta = 0$ in case 4. Such a naming scheme is also present in \cite[][see Table X on page 1070]{DickeyFuller-1981}.

\texttt{urca} package provides critical values for each test in its output. Alternatively, the critical values can be also found in \cite[][see Appendix B, especially the tables B.6 and B.7]{Hamilton-1994} and \cite[][see page 1063]{DickeyFuller-1981}.


Now, let us return to the exercise: Check whether the following time series are stationary.\footnote{The following solutions are heavily based on \cite[][see page 501-504]{Hamilton-1994}}.
\begin{enumerate}[label=--]
    \item \begin{enumerate}[label=(\roman*)]
              \item \verb|CLVMNACSCAB1GQDE.csv|: Real Gross Domestic Product for Germany; from FRED; seasonally adjusted; Quarterly; Millions of Chained 2010 Euros

                    \begin{sol}
                        Due to population growth and technological progress, we would expect positive upward trend in real gdp, as also seen in the figure.

                        \includegraphics[width=\textwidth]{../Figs/real_gdp_tsplot.pdf}

                        The question is whether this trend arises from a unit root process with drift term:
                        \[ H_0: y_t = \alpha + y_{t-1} + u_t, \quad \alpha > 0\]
                        or from a deterministic time trend:
                        \[
                            H_1: y_t = \alpha + \delta t + \rho y_{t-1} + u_t,
                            \quad |\rho| < 1
                        \]

                        Therefore, for the situation here, the test case 4 is suitable.
                    \end{sol}

              \item Transform the \emph{GDP} into \emph{growth rate} using the formula below: \[ g_t = 100 (Y_t / Y_{t-1} - 1) \] where $g_t$ is growth rate, $Y_t$ is seasonally adjusted real GDP.

                    \begin{sol}
                        We would expect GDP growth to move around a constant level, as the figure below also confirms.

                        \includegraphics[width=\textwidth]{../Figs/gdp_growth_tsplot.pdf}

                        The question is whether this arises from a unit root
                        \[ H_0: y_t = y_{t-1} + u_t \]
                        or from a stationary process around a constant
                        \[
                            H_1: y_t = \alpha + \rho y_{t-1} + u_t,
                            \quad |\rho| < 1
                        \]

                        Thus, the case 2 is suitable here.
                    \end{sol}
          \end{enumerate}

    \item \begin{enumerate}[label=(\roman*)]
              \item \verb|LRHUADTTDEQ156S.csv|: Unemployment rate for Germany; from FRED; seasonally adjusted; Quarterly; percent

              \item Transform the \emph{unemployment rate} into the \emph{change in unemployment rate} by taking the first difference.
          \end{enumerate}

    \item \begin{enumerate}[label=(\roman*)]
              \item \verb|CP0000DEM086NEST|: Harmonized Index of Consumer Prices: All Items for Germany; from FRED; Index $2015=100$; Not Seasonally Adjusted; Quarterly - End of Period; Adjusted

              \item Log change in Prices
          \end{enumerate}
\end{enumerate}

\begin{sol}
    \lstinputlisting{../R-files/unit_root.R}
\end{sol}


\Closesolutionfile{ans}

\clearpage
\printbibliography

\end{document}
