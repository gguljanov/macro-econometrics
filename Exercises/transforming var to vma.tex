\begin{enumerate}
    \item Using R, plot the Auto-correlation function (ACF) and the Partial Auto-correlation function (PACF) of the following models, using \verb|ARMAacf(...)|:
          \begin{align*}
              y_t
               & = \theta y_{t-1} + \nu_t,
              \\
              \tilde y_t
               & = \nu_t + \theta \nu_{t-1},
          \end{align*}
          where, $\nu_t \sim N(0, 1)$ and $\theta = 0.45$.

          \begin{sol}
              \lstinputlisting{../R-files/simulate_ar1_ma1.R}
          \end{sol}

    \item Determine the first four coefficients of the \maq[$\infty$]{} representation of the following \arp[1]{} process:
          \[ y_t = 0.25 y_{t-1} + \nu_t \]

          \begin{sol}
              Let us rewrite the above process:
              \begin{align*}
                  y_t              & = 0.25 y_{t-1} + \nu_t \\
                  y_t - 0.25 L y_t & = \nu_t                \\
                  (1 - 0.25L) y_t  & = \nu_t                \\
                  \phi(L) y_t      & = \theta(L) \nu_t
              \end{align*}
              where, $\theta(L) = 1$.

              Under the assumption that all roots of $\phi(z)=0$ are outside the unit circle, we can write the \armapq{} process as \maq[$\infty$]{}.

              Formally, we have
              \begin{align*}
                  y_t & = \frac{\theta(L)}{\phi(L)}\epsilon_t \\
                      & = \psi(L)\epsilon_t
              \end{align*}
              with
              \begin{align*}
                  \psi(L) & = \sum_{j=0}^{\infty}\psi_{j}L^{j}              \\
                          & = (\psi_0+\psi_1 L+\psi_2 L^2+\psi_3 L^3+\dots)
              \end{align*}

              By identification,
              \begin{align*}
                  \frac{\theta(z)}{\phi(z)}
                   & = \psi(z)                                          \\
                   & \Downarrow                                         \\
                  (\psi_0+\psi_1 z+\psi_2 z^2+\psi_3 z^3+\dots) \times ~~~
                   &                                                    \\
                  (1-\phi_1 z -\phi_2 z^2-\dots -\phi_p z^p)
                   & = (1+\theta_1 z +\theta_2 z^2+\dots +\theta_q z^q)
              \end{align*}

              By applying this rule to the process here, we have:
              \begin{align*}
                  (\psi_0 + \psi_1 z + \psi_2 z^2 + \psi_3 z^3 + \dots)
                  (1 - 0.25z)
                   & = 1
                  \\
                   & \Downarrow
                  \\
                  \colorbox{lightgray}{$\psi_0$}
                  + \colorbox{pink}{$\psi_1 z$}
                  + \colorbox{yellow}{$\psi_2 z^2$}
                  + \colorbox{lime}{$\psi_3 z^3$}
                  + \quad \dots \quad
                   &
                  \\
                  \colorbox{pink}{$-0.25 \psi_0 z$}
                  + \colorbox{yellow}{$(-0.25 \psi_1 z^2)$}
                  + \colorbox{lime}{$(-0.25 \psi_2 z^3)$}
                  + (-0.25 \psi_3 z^4)
                  + \quad \dots \quad
                   & =
                  \colorbox{lightgray}{$1$}
                  + \colorbox{pink}{($0 \times z$)}
                  + \colorbox{yellow}{$(0 \times z)$}
                  + \colorbox{lime}{$(0 \times z)$}
              \end{align*}
              \begin{align*}
                  \psi_0            & = 1                                \\
                  \psi_1-0.25\psi_0 & = 0 \Rightarrow \psi_1 = 0.25      \\
                  \psi_2-0.25\psi_1 & = 0 \Rightarrow \psi_2 = 0.0625    \\
                  \psi_3-0.25\psi_2 & = 0 \Rightarrow \psi_3 =  0.015625 \\
                                    & \vdots
              \end{align*}
              For $j \geq 1$,
              \[ \psi_j-0.25 = 0 \]
          \end{sol}

    \item Consider the following VAR(1) process $y_t = \mu + \Phi_1 y_{t-1} + \nu_t$
          \begin{align*}
              \begin{pmatrix} y_{1,t} \\ y_{2,t} \\ y_{3,t} \end{pmatrix}
               & = \begin{pmatrix} 0 \\ 0 \\ 0 \end{pmatrix}
              + \begin{pmatrix}
                    0.5 & 0   & 0   \\
                    0.1 & 0.1 & 0.3 \\
                    0   & 0.2 & 0.3
                \end{pmatrix}
              \begin{pmatrix}
                  y_{1, {t-1}} \\ y_{2, {t-1}} \\ y_{3, {t-1}}
              \end{pmatrix}
              + \begin{pmatrix}
                    \nu_{1,t} \\ \nu_{2,t} \\ \nu_{3,t}
                \end{pmatrix}
          \end{align*}
          provided that $\nu_t \sim N(0, V)$ and
          \[
              V
              = \begin{pmatrix}
                  2.25 & 0 & 0 \\ 0 & 1 & 0.5 \\ 0 & 0.5 & 0.74
              \end{pmatrix}
          \]
          Compute the coefficients $\Psi_0, \Psi_1, \dots \in \mathbb{R}^{3\times 3}$ of the lag polynomial $\Psi(L) := \sum_{i=0}^\infty \Psi_i L^i$, and a $c \in \mathbb{R}^3$ such that\footnote{Taken from \cite[][See section 16, "Understanding multivariate time series concepts"]{Mutschler-2018-github_repo}.}
          \begin{align*}
              y_t = c + \Psi(L) \nu_t
          \end{align*}

          \begin{sol}
              We will use the very efficient method of matching coefficients to transform the \varp[1]{} model into a \vmaq[$\infty$]{} representation.
              \begin{align*}
                  y_t
                   & = \Phi_1 y_{t-1} + \nu_t
                  \\
                  \underbrace{(I_3 - \Phi_1 L)}_{\Phi(L)} y_t
                   & = \nu_t
                  \\
                  \Phi(L) y_t
                   & = \nu_t
                  \\
                  \Phi(L)^{-1} \Phi(L) y_t
                   & = \Phi(L)^{-1} \nu_t
                  \\
                  y_t
                   & = \Phi(L)^{-1} \nu_t
                  = c + \Psi(L) \nu_t
                  \\
                  \Rightarrow
                   & c = 0,
                  \Psi(L) = \Phi(L)^{-1}
              \end{align*}
              In the method of matching coefficient we compare the coefficient matrices multiplied to each power of $L$. That is, the expression on the left hand side has to match the expression on the right hand side. In our case:
              \begin{align*}
                  \Psi(L)
                   & = \Phi(L)^{-1}
                  \\
                  \Phi(L) \Psi(L)
                   & = I_3
                  \\
                  (I_3 - \Phi_1 L) \left(\sum_{i=0}^\infty \Psi_i L^i\right)
                   & = I_3
              \end{align*}

              Expanding the two brackets:
              \begin{align*}
                  \Psi_0 L^0
                   & + \Psi_1 L^1 + \Psi_2 L^2 + \dots
                  \\
                   & - \Phi_1 \Psi_0 L^1 - \Phi_1 \Psi_1 L^2 - \dots = I_3 L^0
              \end{align*}

              Finally, let us compare the expressions on the left hand side to those on the right hand side:
              \begin{align}
                  L^0
                                    & : \Phi_0 = I_3
                  \Rightarrow
                  \Phi_0 = I_3
                  = \begin{pmatrix}
                        1 & 0 & 0 \\ 0 & 1 & 0 \\ 0 & 0 & 1
                    \end{pmatrix}
                  \nonumber                                       \\
                  L^1
                                    & : \Phi_1 - A \Phi_0 = 0
                  \Rightarrow
                  \Phi_1 = A \Phi_0
                  = A
                  = \begin{pmatrix}
                        0.5 & 0 & 0 \\ 0.1 & 0.1 & 0.3 \\ 0 & 0.2 & 0.3
                    \end{pmatrix}
                  \nonumber                                       \\
                  L^2
                                    & : \Phi_2 - A \Phi_1 = 0
                  \Rightarrow
                  \Phi_2 = A \Phi_1
                  = A^2
                  = \begin{pmatrix}
                        0.25 & 0 & 0 \\ 0.06 & 0.07 & 0.12 \\ 0.02 & 0.08 & 0.15
                    \end{pmatrix}
                  \nonumber                                       \\
                                    & \vdots
                  \nonumber                                       \\
                  \text{In general} & : \Phi_s = A^s \label{allg}
              \end{align}
          \end{sol}
\end{enumerate}