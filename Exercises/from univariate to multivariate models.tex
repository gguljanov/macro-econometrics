\begin{enumerate}
    \item Consider the following two \armapq[1, 1]{} models, for inflation and output, respectively:
          \[ \pi_t = 0.3 + 0.8 \pi_{t-1} + \epsilon_t - 0.1 \epsilon_{t-1} \]
          and
          \[ o_t = -0.64 o_{t-1} + \eta_t + 0.2 \eta_{t-1}. \]

          Write these two \armapq[1, 1]{} models as \varmapq[1, 1]{}.

          \begin{sol}
              \begin{align*}
                  \underbrace{
                      \begin{pmatrix}
                          \pi_t \\ o_t
                      \end{pmatrix}
                  }_{
                      y_t
                  }
                  = \underbrace{
                      \begin{pmatrix} 0.3 \\ 0 \end{pmatrix}
                  }_{
                      \mu
                  }
                  + \underbrace{
                      \begin{bmatrix}
                          0.8 & 0    \\
                          0   & -0.6
                      \end{bmatrix}
                  }_{
                      \Phi_1
                  }
                  \underbrace{
                      \begin{pmatrix}
                          \pi_{t-1} \\ o_{t-1}
                      \end{pmatrix}
                  }_{
                  y_{t-1}
                  }
                  + \underbrace{
                      \begin{pmatrix}
                          \epsilon_{t} \\ \eta_{t}
                      \end{pmatrix}
                  }_{
                      \nu_t
                  }
                  + \underbrace{
                      \begin{bmatrix}
                          -0.1 & 0   \\
                          0    & 0.2
                      \end{bmatrix}
                  }_{
                      \Psi_1
                  }
                  \underbrace{
                      \begin{pmatrix}
                          \epsilon_{t-1} \\ \eta_{t-1}
                      \end{pmatrix}
                  }_{
                  \nu_{t-1}
                  }
              \end{align*}
          \end{sol}

    \item Consider a \varp[1]{} model with the following parameter matrix
          \begin{align*}
              \mu    & = \begin{pmatrix} 0 \\ -0.04 \end{pmatrix}                    \\
              \Phi_1 & = \begin{bmatrix} -0.81 & 0.26 \\ -0.34 & -1.53 \end{bmatrix}
          \end{align*}

          Write the above \varp[1]{} model as two univariate models. Are the resulting univariate models \arp[1]{}?

          \begin{sol}
              Let us first write the \varmapq[1, 1]{} model more clearly:
              \begin{align*}
                  \begin{pmatrix} x_{t} \\ z_{t} \end{pmatrix}
                  = \begin{pmatrix} 0 \\ -0.04 \end{pmatrix}
                  + \begin{bmatrix} -0.81 & 0.26 \\ -0.34 & -1.53 \end{bmatrix}
                  \begin{pmatrix} x_{t-1} \\ z_{t-1} \end{pmatrix}
                  + \begin{pmatrix} \epsilon_{t} \\ \eta_{t} \end{pmatrix}
              \end{align*}

              The two univariate models would then be as follows:
              \begin{align*}
                  x_t & = -0.81 x_{t-1} + 0.26 z_{t-1} + \epsilon_t    \\
                  z_t & = -0.04 - 0.34 x_{t-1} - 1.53 z_{t-1} + \eta_t
              \end{align*}

              The univariate models above are not exactly \arp[1]{} model, but are rather, Autoregressive Distributed Lag models, because they contain not only their own lags but also lags of the other variable.
          \end{sol}

    \item Why are we concerned with multivariate time series? For example, why do we model the VAR(1) process
          \begin{align*}
              \begin{pmatrix} y_{1,t} \\ y_{2,t} \end{pmatrix}
              = \begin{pmatrix} \mu_1 \\ \mu_2 \end{pmatrix}
              + \begin{pmatrix}
                    \phi_{11} & \phi_{12} \\
                    \phi_{21} & \phi_{22}
                \end{pmatrix}
              \begin{pmatrix} y_{1, {t-1}} \\ y_{2, {t-1}} \end{pmatrix}
              + \begin{pmatrix} \nu_{1, t} \\ \nu_{2, t} \end{pmatrix}
          \end{align*}
          simultaneously, instead of two models for each variable separately?\footnote{Taken from \cite[][see the section 16, titled ``Understanding multivariate time series concepts"]{Mutschler-2018-github_repo}}

          \begin{sol}
              For the specification of multi-equation models, we require a clear distinction between exogenous and endogenous variables. In economic theory, this is often not clear or arbitrarily made in practice. Vectorautoregressive models do not need this distinction, they can rather be understood as a dynamic version of a simultaneous multi-equation model. This corresponds to reality, because economic variables are generated by dynamic processes, and often are interdependent.

              Therefore, VAR models provide a powerful instrument. They also take into account things like non-stationarity (cointegration and long-term equilibria), as well as the analysis of dynamics of random shocks / impulses.

              An Example: we are also able to consider correlations between $\nu_{1, t}$ and $\nu_{2, t}$.

              Lastly, VAR models tend to have better predictive power than multi-equation models and are often used as a benchmark for different forecasting models.
          \end{sol}

    \item Consider the following \svarp[1]{} model:
          \begin{align*}
              \underbrace{
                  \begin{bmatrix}
                      b_{0, 1, 1} & b_{0, 1, 2} \\
                      b_{0, 2, 1} & b_{0, 2, 2}
                  \end{bmatrix}
              }_{
                  B_0
              } y_t
              = \underbrace{
                  \begin{bmatrix}
                      b_{1, 1, 1} & b_{1, 1, 2} \\
                      b_{1, 2, 1} & b_{1, 2, 2}
                  \end{bmatrix}
              }_{B_1} y_{t-1}
              + u_t
          \end{align*}

          Write the above model as \varp[1]{}.

          \begin{sol}
              Let us first invert the $B_0$ matrix:
              \begin{align*}
                  B_0^{-1} & = \underbrace{
                      \frac{1}{
                          b_{0, 1, 1} b_{0, 2, 2} - b_{0, 1, 2} b_{0, 2, 1}
                      }
                  }_{:= c}
                  \begin{bmatrix}
                      b_{0, 2, 2}  & -b_{0, 1, 2} \\
                      -b_{0, 2, 1} & b_{0, 1, 1}
                  \end{bmatrix} \\
                           & =
                  \begin{bmatrix}
                      c b_{0, 2, 2}  & -c b_{0, 1, 2} \\
                      -c b_{0, 2, 1} & c b_{0, 1, 1}
                  \end{bmatrix}
              \end{align*}

              \begin{align*}
                  B_0 y_t & = B_1 y_{t-1} + u_t
                  \\[0.4cm]
                  y_t     & = B_0^{-1} B_1 y_{t-1} + B_0^{-1} u_t
                  \\[0.4cm]
                          & = \begin{bmatrix} % B_0^{-1}
                                  c b_{0, 2, 2}  & -c b_{0, 1, 2} \\
                                  -c b_{0, 2, 1} & c b_{0, 1, 1}
                              \end{bmatrix}
                  \begin{bmatrix} % B_1
                      b_{1, 1, 1} & b_{1, 1, 2} \\
                      b_{1, 2, 1} & b_{1, 2, 2}
                  \end{bmatrix}
                  y_{t-1}
                  +
                  \begin{bmatrix} % B_0^{-1}
                      c b_{0, 2, 2}  & -c b_{0, 1, 2} \\
                      -c b_{0, 2, 1} & c b_{0, 1, 1}
                  \end{bmatrix}
                  \begin{pmatrix}
                      u_{1, t} \\
                      u_{2, t}
                  \end{pmatrix}
                  \\[0.4cm]
                          & =
                  \underbrace{
                      \begin{bmatrix} % Phi_1
                          c b_{0, 2, 2} b_{1, 1, 1}
                          - c b_{0, 1, 2} b_{1, 2, 1}
                           & c b_{0, 2, 2} b_{1, 1, 2}
                          - c b_{0, 1, 2} b_{1, 2, 2}
                          \\
                          -c b_{0, 2, 1} b_{1, 1, 1}
                          + c b_{0, 1, 1} b_{1, 2, 1}
                           & -c b_{0, 2, 1} b_{1, 1, 2}
                          + c b_{0, 1, 1} b_{1, 2, 2}
                      \end{bmatrix}
                  }_{\Phi_1}
                  y_{t-1}
                  +
                  \underbrace{
                      \begin{bmatrix} % nu_t
                          c b_{0, 2, 2} u_{1, t} - c b_{0, 1, 2} u_{2, t} \\
                          -c b_{0, 2, 1} u_{1, 6} + c b_{0, 1, 1} u_{2, t}
                      \end{bmatrix}
                  }_{\nu_t}
                  \\[0.4cm]
                          & = \Phi_1 y_{t-1} + \nu_t
              \end{align*}
          \end{sol}

    \item Using R, simulate and plot $250$ observations from the following \varp[1]{} model:
          \[ y_t = \Phi_1 y_{t-1} + \nu_t \]
          where,
          \begin{align*}
              y_0    & = \begin{pmatrix}
                             0 \\
                             0
                         \end{pmatrix}    \\[0.5cm]
              \Phi_1 & = \begin{bmatrix}
                             -0.66 & 0.26  \\
                             -0.30 & -0.58
                         \end{bmatrix}    \\[0.5cm]
              \nu_t  & \sim N(0, I_{2, 2})
          \end{align*}

          \begin{sol}
              \lstinputlisting{../R-files/simulate_var1_250.R}
          \end{sol}

    \item Using R, simulate and plot $250$ observations from the following \svarp[1]{} models:
          \[ B_0 y_t = B_1 y_{t-1} + u_t \]
          where,
          \begin{align*}
              y_0 & = \begin{pmatrix}
                          0 \\
                          0
                      \end{pmatrix}    \\[0.5cm]
              B_0 & = \begin{bmatrix}
                          -0.98 & -0.42 \\
                          0.49  & 1.11
                      \end{bmatrix}    \\[0.5cm]
              B_1 & = \begin{bmatrix}
                          -0.66 & 0.26  \\
                          -0.30 & -0.58
                      \end{bmatrix}    \\[0.5cm]
              u_t & \sim N(0, I_{2, 2})
          \end{align*}

          \begin{sol}
              \lstinputlisting{../R-files/simulate_svar1_250.R}
          \end{sol}
\end{enumerate}
