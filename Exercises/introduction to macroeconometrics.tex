Broadly define the term and research topics of \emph{Macroeconometrics}.

\begin{sol}
    Definition of term:
    \begin{enumerate}[label=]
        \item Combination of modern theoretical macroeconomics (the study of aggregated variables such as economic growth, unemployment and inflation by means of structural macroeconomic models) and econometric methods (the application of formal statistical methods in empirical economics).
    \end{enumerate}

    Research topics:
    \begin{enumerate}[label=\roman*.]
        \item How to identify sources of fluctuations, e.g. how important are monetary policy shocks as opposed to other shocks for movements in aggregate output? [forecast error variance decomposition]

        \item Understand propagation of shocks, e.g. what happens to aggregate hours worked over the next two years in response to a technology shock in the current quarter? [impulse response function]

        \item Forecasting the future, e.g. how will inflation and output growth rates evolve over next eight quarters. [forecasting]

        \item Predict effect of policy changes, e.g. how will output and inflation respond to an unanticipated change in nominal interest rate? [impulse response function and forecast scenarios]

        \item Structural changes in the economy, e.g. has monetary policy changed in the early 1980s, why did volatility of many macroeconomic time series drop in the mid 1980s, [historical decomposition]

        \item How much of the recession of 1982  would have deepened had monetary policymakers not responded to output growth at all. [policy counterfactual]
    \end{enumerate}
\end{sol}