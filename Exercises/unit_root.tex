In our previous exercises, we plotted the data series and qualitatively judged whether they look stationary or not. In this exercise, we will check them formally using ADF test.

Following \cite[][see chapter 17]{Hamilton-1994}, ADF-test can be summarized as follows:

\begin{tabular}{r|llll}
     & Estimated process
     & True process
     & $H_0$
     & Test stat.
    \\
    \toprule
    Case 1:
     & $y_t = \rho y_{t-1} + u_t$
     & $y_t = y_{t-1} + u_t$
     & $\rho = 1$
     & $\tau_1$
    \\
    Case 2:
     & $y_t = \alpha + \rho y_{t-1} + u_t$
     & $y_t = y_{t-1} + u_t$
     & $\rho = 1$ (and $\alpha = 0$)
     & $\tau_2$ (and $\Phi_1$)
    \\
    Case 3:
     & $y_t = \alpha + \rho y_{t-1} + u_t$
     & $y_t = \alpha + y_{t-1} + u_t$; $\alpha \neq 0$
     & $\rho = 1$
     &
    \\
    Case 4:
     & $y_t = \alpha + \rho y_{t-1} + \delta t +  u_t$
     & $y_t = \alpha + y_{t-1} + u_t$; $\alpha$ any
     & $\rho = 1$ (and $\delta = 0$)
     & $\tau_3$ (and $\Phi_3$)
\end{tabular}

The asymptotic properties of the \ols{} estimate $\hat\rho$ when the true value of $\rho$ is unity, depend on which case we want to use.

Which is the "correct case" to use? The answer depends on why we are interested in testing for a unit root:
\begin{enumerate}[label=(\arabic*)]
    \item If the analyst has a specific null Hypothesis about the data generating process, then this would guide the choice of the test.

    \item If there is no such a guidance, a general principle would be to fit a specification that is plausible description of the data under null and alternative Hypotheses.
\end{enumerate}

For analyzing unit roots, we will use an r package called \texttt{urca}, from \cite{Pfaff-2008}. The names in the "Test stat." column of the table above, are used in the outputs generated by the commands of this package. $\tau_1$, $\tau_2$, $\tau_3$ refer to tests of only $\rho = 1$ in each respective case. The joint test are called $\Phi_1$, $\Phi_2$, and $\Phi_3$. $\Phi_2$ is the joint test of $\alpha = 0$, $\rho = 1$, and $\delta = 0$ in case 4. Such a naming scheme is also present in \cite[][see Table X on page 1070]{DickeyFuller-1981}.

\texttt{urca} package provides critical values for each test in its output. Alternatively, the critical values can be also found in \cite[][see Appendix B, especially the tables B.6 and B.7]{Hamilton-1994} and \cite[][see page 1063]{DickeyFuller-1981}.


Now, let us return to the exercise: Check whether the following time series are stationary.\footnote{The following solutions are heavily based on \cite[][see page 501-504]{Hamilton-1994}}.
\begin{enumerate}[label=--]
    \item \begin{enumerate}[label=(\roman*)]
              \item \verb|CLVMNACSCAB1GQDE.csv|: Real Gross Domestic Product for Germany; from FRED; seasonally adjusted; Quarterly; Millions of Chained 2010 Euros

                    \begin{sol}
                        Due to population growth and technological progress, we would expect positive upward trend in real gdp, as also seen in the figure.

                        \includegraphics[width=\textwidth]{../Figs/real_gdp_tsplot.pdf}

                        The question is whether this trend arises from a unit root process with drift term:
                        \[ H_0: y_t = \alpha + y_{t-1} + u_t, \quad \alpha > 0\]
                        or from a deterministic time trend:
                        \[
                            H_1: y_t = \alpha + \delta t + \rho y_{t-1} + u_t,
                            \quad |\rho| < 1
                        \]

                        Therefore, for the situation here, the test case 4 is suitable.
                    \end{sol}

              \item Transform the \emph{GDP} into \emph{growth rate} using the formula below: \[ g_t = 100 (Y_t / Y_{t-1} - 1) \] where $g_t$ is growth rate, $Y_t$ is seasonally adjusted real GDP.

                    \begin{sol}
                        We would expect GDP growth to move around a constant level, as the figure below also confirms.

                        \includegraphics[width=\textwidth]{../Figs/gdp_growth_tsplot.pdf}

                        The question is whether this arises from a unit root
                        \[ H_0: y_t = y_{t-1} + u_t \]
                        or from a stationary process around a constant
                        \[
                            H_1: y_t = \alpha + \rho y_{t-1} + u_t,
                            \quad |\rho| < 1
                        \]

                        Thus, the case 2 is suitable here.
                    \end{sol}
          \end{enumerate}

    \item \begin{enumerate}[label=(\roman*)]
              \item \verb|LRHUADTTDEQ156S.csv|: Unemployment rate for Germany; from FRED; seasonally adjusted; Quarterly; percent

              \item Transform the \emph{unemployment rate} into the \emph{change in unemployment rate} by taking the first difference.
          \end{enumerate}

    \item \begin{enumerate}[label=(\roman*)]
              \item \verb|CP0000DEM086NEST|: Harmonized Index of Consumer Prices: All Items for Germany; from FRED; Index $2015=100$; Not Seasonally Adjusted; Quarterly - End of Period; Adjusted

              \item Log change in Prices
          \end{enumerate}
\end{enumerate}

\begin{sol}
    \lstinputlisting{../R-files/unit_root.R}
\end{sol}