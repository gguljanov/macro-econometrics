\begin{labeling}{argument}
    \item[\textbf{Matrix:}] A rectangular array of real numbers:
    \[
        \begin{bmatrix}
            a_{11} & a_{12} & \ldots a_{1n} \\
            a_{21} & a_{22} & \ldots a_{2n} \\
            \vdots & \vdots & \vdots        \\
            a_{m1} & a_{m2} & \ldots a_{mn}
        \end{bmatrix}
    \]

    For the following definitions, we use $m \times n$ matrices $A$ and $B$. When we talk about square matrices, we assume $m = n$.

    \item[\textbf{Addition:}] Adding two matrices is adding the respective elements:
    \[ A + B = (a_{ij}) + (b_{ij}) = (a_{ij} + b_{ij}). \]
    Note that the dimensions of the summands must be the same.

    \textbf{Exercise:} Do the following addition:
    \[
        \begin{bmatrix}
            1  & 12 & 8  \\
            15 & 6  & 10
        \end{bmatrix}
        +
        \begin{bmatrix}
            3   & 2 & 16  \\
            -14 & 7 & -11
        \end{bmatrix}
        =
        ?
    \]

    \begin{sol}
        \[
            \begin{bmatrix}
                4 & 14 & 24 \\
                1 & 13 & -1
            \end{bmatrix}
        \]
    \end{sol}


    \item[\textbf{Multiplication:}] Multiply each row of the first matrix with each column of the second matrix, element-wise and take sum. Order of the row and column gives the place of the sum in the resulting matrix:
    \[
        AB = \bigg( \sum_{j=1}^n a_{ij} b_{jk} \bigg)
    \]

    \textbf{Exercise:} Do the following multiplication:
    \[
        \begin{bmatrix}
            2  & -7 \\
            -9 & 5  \\
            4  & 3
        \end{bmatrix}
        \times
        \begin{bmatrix}
            -6 & 1 \\
            3  & 8
        \end{bmatrix}
        =
        ?
    \]

    \begin{sol}
        \[
            \begin{bmatrix}
                2 \times (-6) + (-7) \times 3
                 & 2 \times 1 + (-7) \times 8 \\
                -9 \times (-6) + 5 \times 3
                 & -9 \times 1 + 5 \times 8   \\
                4 \times (-6) + 3 \times 3
                 & 4 \times 1 + 3 \times 8
            \end{bmatrix}
            =
            \begin{pmatrix}
                -33 & -54 \\
                69  & 31  \\
                -15 & 28
            \end{pmatrix}
        \]
    \end{sol}

    \item[\textbf{Transpose:}] Turning rows into columns or vice versa:
    \[
        (a'_{ij}) = (a_{ji})
    \]
    It is denoted as $A'$.

    \textbf{Exercise:} Take the transpose of the following matrix

    \[
        A = \begin{bmatrix}
            2  & -7 \\
            -9 & 5  \\
            4  & 3
        \end{bmatrix}
    \]

    \begin{sol}
        \[
            A' = \begin{bmatrix}
                2  & -9 & 4 \\
                -7 & 5  & 3
            \end{bmatrix}
        \]
    \end{sol}

    \item[\textbf{Square Matrix:}] The number of rows of a square matrix is equal to its number of columns.

    \textbf{Exercise:} Is the following matrix square?
    \[
        A = \begin{bmatrix}
            2  & -7 \\
            -9 & 5
        \end{bmatrix}
    \]

    \begin{sol}
        Yes, it is. Because, it has two rows and two columns.
    \end{sol}

    \item[\textbf{Rank:}] Number of linearly independent rows/columns is called row/column rank. A set of vectors $x_1, x_2, \ldots x_n$ is set to be linearly independent, if there exists no set of scalars $\alpha_1, \alpha_2 \ldots \alpha_n$, not all zero, such that $\sum \alpha_i x_i = 0$.\footnote{\cite[][see page 8]{MagnusNeudecker-2019-part1}} The rank is denoted as $r(A)$.

    \textbf{Exercise:} Find the rank of the following matrix using R:

    \[
        A = \begin{bmatrix}
            2  & -7 \\
            -9 & 5  \\
            4  & 3
        \end{bmatrix}
    \]

    \begin{sol}
        \lstinputlisting{../R-files/matrix_rank.R}

        The R-script tells that the rank of matrix $A$ is two, i.e. $r(A) = 2$. Note that the rank of a matrix can be at most $min(m, n)$, where $m$ is number of rows and $n$ is number of columns. Here, we have three rows and two columns. So, the rank can be at most two.
    \end{sol}

    \item[\textbf{Inverse:}] When you multiply a matrix with its inverse, you get an identity matrix:
    \[
        A A^{-1} = A^{-1} A = I_n
    \]
    Note that matrix $A$ must be square matrix. So, here $m = n$ for matrix $A$. Moreover, it might be that an inverse does not exist. The inverse is denoted as $A^{-1}$.

    \textbf{Exercise:} Take the inverse of the following matrix and using R, check, if your answer is correct by multiplying the inverse with the original matrix:
    \[
        A = \begin{bmatrix}
            2  & -7 \\
            -9 & 5
        \end{bmatrix}
    \]

    \begin{sol}
        For two-by-two matrices, there is a simple formula for inverting:
        \[
            \begin{bmatrix}
                -0.09433962 & -0.13207547 \\
                -0.16981132 & -0.03773585
            \end{bmatrix}
        \]
    \end{sol}

    \item[\textbf{Determinant:}] Determinant is a special number associated with a square matrix. In R, you can find the determinan of matrix $A$ with \verb|determinant(A, logarithm = FALSE)|. The determinant is denoted as $|A|$.

    \textbf{Exercise:} Find the determinant of the following matrix:
    \[
        A = \begin{bmatrix}
            1  & 12 & 8  & 13 \\
            15 & 6  & 10 & 3  \\
            14 & 7  & 11 & 2  \\
            4  & 9  & 5  & 16 \\
        \end{bmatrix}
    \]

    \begin{sol}
        \lstinputlisting{../R-files/matrix_determinant.R}
    \end{sol}

    \item[\textbf{Trace:}] Sum of a diagonal elements of a square matrix. It is denoted as $tr(A)$.

    \textbf{Exercise:} Find the trace of the following matrix.
    \[
        A = \begin{bmatrix}
            1  & 12 & 8  & 13 \\
            15 & 6  & 10 & 3  \\
            14 & 7  & 11 & 2  \\
            4  & 9  & 5  & 16 \\
        \end{bmatrix}
    \]

    \begin{sol}
        $tr(A) = 1 + 6 + 11 + 16 = 34$
    \end{sol}

    \item[\textbf{Eigenvalues:}] The eigenvalues of $A$ are defined as the roots of the characteristic equation:\footnote{This definition is from \cite[][page 12]{MagnusNeudecker-2019-part1}}
    \[ |\lambda I_n - A| = 0 \]

    \item[\textbf{Eigenvectors:}] Let $\lambda$ be an eigenvalue of $A$. Then, there exists a vector $x$ ($x \neq 0$) such that
    \[ (\lambda I - A) x = 0. \]
    The vector $x$ is called a (column) eigenvector of $A$ associated with the eigenvalue $\lambda$.\footnote{This definition is from \cite[][page 12]{MagnusNeudecker-2019-part1}}

    \textbf{Exercise:} Using R, determine the eigenvalues and eigenvectors of the following matrix:
    \[
        A = \begin{bmatrix}
            0.5 & 0   & 0   \\
            0.1 & 0.1 & 0.3 \\
            0   & 0.2 & 0.3
        \end{bmatrix}
    \]

    \begin{sol}
        \lstinputlisting{../R-files/eigen_values_vectors.R}
    \end{sol}

    \item[\textbf{Vectorization:}] Stacks the columns of a matrix one after another, vertically, resulting in a long column vector:
    \begin{align*}
        A
         & = \begin{bmatrix}
                 a_{11} & a_{12} & a_{13} \\
                 a_{21} & a_{22} & a_{23} \\
                 a_{31} & a_{32} & a_{33}
             \end{bmatrix}
        \\
        vec(A)
         & = \begin{pmatrix}
                 a_{11} \\ a_{21} \\ a_{31} \\
                 a_{21} \\ a_{22} \\ a_{23} \\
                 a_{31} \\ a_{32} \\ a_{33}
             \end{pmatrix}
    \end{align*}

    \textbf{Exercise:} Determine the following:
    \begin{align*}
        vec \begin{bmatrix}
                1 & 3 & 2 \\
                1 & 0 & 0 \\
                1 & 2 & 2
            \end{bmatrix}
        =
        ?
    \end{align*}

    \textbf{Exercise:} Determine the following:
    \begin{align*}
        vec \begin{pmatrix}
                1 & 3 & 2 \\
                1 & 0 & 0 \\
                1 & 2 & 2
            \end{pmatrix}
        =
        ?
    \end{align*}

    \begin{sol}
        \begin{align*}
            vec \begin{pmatrix}
                    1 & 3 & 2 \\
                    1 & 0 & 0 \\
                    1 & 2 & 2
                \end{pmatrix}
            =
            \begin{pmatrix}
                1 \\ 1 \\ 1 \\ 3 \\ 0 \\ 2 \\ 2 \\ 0 \\ 2
            \end{pmatrix}
        \end{align*}
    \end{sol}

    \item[\textbf{Half-vectorization:}] Similar to $vec(\cdot)$ operator, but considers only those elements on or below the diagonal:\footnote{for more details, see \cite[][page 300-301]{Hamilton-1994}}
    \begin{align*}
        A
         & = \begin{bmatrix}
                 a_{11} & a_{12} & a_{13} \\
                 a_{21} & a_{22} & a_{23} \\
                 a_{31} & a_{32} & a_{33}
             \end{bmatrix}
        \\
        vech(A)
         & = \begin{pmatrix}
                 a_{11} \\ a_{21} \\ a_{31} \\ a_{22} \\ a_{23} \\ a_{33}
             \end{pmatrix}
    \end{align*}

    \textbf{Exercise:} Determine the following:
    \begin{align*}
        vech \begin{pmatrix}
                 1 & 3 & 2 \\
                 1 & 0 & 0 \\
                 1 & 2 & 2
             \end{pmatrix}
        =
        ?
    \end{align*}

    \begin{sol}
        \begin{align*}
            vec \begin{pmatrix}
                    1 & 3 & 2 \\
                    1 & 0 & 0 \\
                    1 & 2 & 2
                \end{pmatrix}
            =
            \begin{pmatrix}
                1 \\ 1 \\ 1 \\ 0 \\ 2 \\ 2
            \end{pmatrix}
        \end{align*}
    \end{sol}

    \item[\textbf{Kronecker Product:}] Multiplies each element of the first matrix with the entire second matrix:
    \[
        A \otimes B = \begin{bmatrix}
            a_{11} B & \cdots & a_{1n} B \\
            \vdots   &        & \vdots   \\
            a_{m1} B & \cdots & a_{nn} B \\
        \end{bmatrix}
    \]

    \textbf{Exercise:}
    \[
        \underbrace{%
            \begin{bmatrix}
                1 & 3 & 2 \\
                1 & 0 & 0 \\
                1 & 2 & 2
            \end{bmatrix}%
        }_{3 \times 3}
        \otimes
        \underbrace{%
            \begin{bmatrix}
                0 & 5 \\
                5 & 0 \\
                1 & 1
            \end{bmatrix}%
        }_{3 \times 2}
        = ?
    \]

    \begin{sol}
        \begin{align*}
            \underbrace{%
                \begin{bmatrix}
                    1 & 3 & 2 \\
                    1 & 0 & 0 \\
                    1 & 2 & 2
                \end{bmatrix}%
            }_{3 \times 3}
            \otimes
            \underbrace{%
                \begin{bmatrix}
                    0 & 5 \\
                    5 & 0 \\
                    1 & 1
                \end{bmatrix}%
            }_{3 \times 2}
            =
            \underbrace{%
                \begin{bmatrix}
                    1\cdot\begin{pmatrix}
                              0 & 5 \\
                              5 & 0 \\
                              1 & 1
                          \end{pmatrix}
                     & 3\cdot\begin{pmatrix}
                                 0 & 5 \\
                                 5 & 0 \\
                                 1 & 1
                             \end{pmatrix}
                     & 2\cdot\begin{pmatrix}
                                 0 & 5 \\
                                 5 & 0 \\
                                 1 & 1
                             \end{pmatrix}
                    \\[0.8cm]
                    1\cdot\begin{pmatrix}
                              0 & 5 \\
                              5 & 0 \\
                              1 & 1
                          \end{pmatrix}
                     & 0\cdot\begin{pmatrix}
                                 0 & 5 \\
                                 5 & 0 \\
                                 1 & 1
                             \end{pmatrix}
                     & 0\cdot\begin{pmatrix}
                                 0 & 5 \\
                                 5 & 0 \\
                                 1 & 1
                             \end{pmatrix}
                    \\[0.8cm]
                    1\cdot\begin{pmatrix}
                              0 & 5 \\
                              5 & 0 \\
                              1 & 1
                          \end{pmatrix}
                     & 2\cdot\begin{pmatrix}
                                 0 & 5 \\
                                 5 & 0 \\
                                 1 & 1
                             \end{pmatrix}
                     & 2\cdot\begin{pmatrix}
                                 0 & 5 \\
                                 5 & 0 \\
                                 1 & 1
                             \end{pmatrix}
                \end{bmatrix}%
            }_{9 \times 6}
        \end{align*}
    \end{sol}
\end{labeling}