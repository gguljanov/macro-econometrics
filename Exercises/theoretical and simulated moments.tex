\begin{enumerate}
\item Derive the theoretical mean ($\mu_y$), covariance matrix ($\Gamma(0)$) and autocovariance matrix ($\Gamma(h)$) of the covariance stationary \varp[1]{} model: $y_t = \mu + \Phi y_{t-1} + \nu_t, \quad \nu_t \sim N(0, V)$

\begin{sol}
    \textbf{Mean:} \\
    \begin{align*}
        \underbrace{\E(y_t)}_{\mu_y}
        & = \underbrace{\E(\mu)}_{\mu}
        + \underbrace{\E(\Phi y_{t-1})}_{\Phi \mu_y}
        + \underbrace{\E(\nu_t)}_{0}
        \\
        \Leftrightarrow \mu_y      
        & = (I - \Phi)^{-1} \mu
    \end{align*}
    Put $\mu = (I - \Phi) \mu_y:
    \underbrace{y_t -\mu_y}_{ \tilde{y}_{t}}
    = \Phi (\underbrace{y_{t-1} - \mu_y}_{\tilde{y}_{t-1}})
    + \nu_t$.


    \textbf{Covariance matrix and Autocovariance matrix:} \\
    Note: $\Gamma_y(h) = \Gamma_{\tilde{y}}(h)$
    \begin{align*}
        \Gamma_y(0)
        & = \E\left[\tilde{y}_t \tilde{y}_t'\right]
        = \E\left[(\Phi \tilde{y}_{t-1} + \nu_t)(\Phi \tilde{y}_{t-1} + \nu_t)'\right]
        \\
        & = \Phi \underbrace{\E\left[\tilde{y}_{t-1}\right]}_{\Gamma_y(0)} \Phi'
        + \Phi \underbrace{\E\left[\tilde{y}_{t-1}\nu_t'\right]}_{0}
        + \underbrace{\E\left[\nu_t \tilde{y}_{t-1}'\right]}_{0} \Phi'
        + \underbrace{\E\left[\nu_t \nu_t'\right]}_{V}
        \\
        \Leftrightarrow \Gamma_y(0)
        & = \Phi \Gamma_y(0) \Phi' + V
    \end{align*}
    This is a discrete Lyapunov matrix equation, see exercise 1.5:
    \[
        vec\bigg( \Gamma_y(0) \bigg)
        = \bigg( I - \Phi \otimes \Phi \bigg)^{-1} vec\bigg( \Sigma_u \bigg)
    \]
    \begin{align*}
        \Gamma_y(1)
        & = \E\left[\tilde{y}_t \tilde{y}_{t-1}'\right]
        = \E\left[(\Phi \tilde{y}_{t-1} + \nu_t)\tilde{y}_{t-1}'\right]
        = \Phi \E\underbrace{
            \left[\tilde{y}_{t-1}\tilde{y}_{t-1}'\right]
        }_{\Gamma_y(0)}
        + \underbrace{\E\left[\nu_t y_{t-1}'\right]}_{0}
        = \Phi \Gamma_y(0)
        \\
        \Gamma_y(2)
        & =  \E\left[\tilde{y}_t \tilde{y}_{t-2}'\right]
        = \E\left[(\Phi \tilde{y}_{t-1} + \nu_t)\tilde{y}_{t-2}'\right]
        = \Phi \Gamma_y(1) = \Phi^2 \Gamma_y(0)
        \\
        \Gamma_y(h)
        & = \Phi^h \Gamma_y(0)
    \end{align*}
\end{sol}

\item Simulate $R = 100$ datasets each with $T = 100$ observations for
\begin{align*}
    y_t = \begin{pmatrix}0.2 &0.3 \\-0.6 & 1.1 \end{pmatrix} y_{t-1}  + \nu_t
\end{align*}
provided that $\nu_t \sim N(0, V)$ and $V = \begin{pmatrix}
    0.9 & 0.2 \\ 0.2 & 0.5
\end{pmatrix}$.

Compute the sample mean and sample covariance matrix. Compare the results with parts (1) and (2). How does your choice of $R$ or $T$ change results?

\begin{sol}
  \lstinputlisting{../R-files/var1_simulate_moments.R}
    The Monte Carlo shows that either increasing $R$ or $T$ (or both) captures the moments better.
\end{sol}
\end{enumerate}