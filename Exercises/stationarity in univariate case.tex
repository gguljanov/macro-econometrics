\begin{enumerate}
    \item Show that an \maq[1]{} process, $y_t = \nu_t + \psi \nu_{t-1}$ with $\nu_t \sim N(0, 1)$, is covariance-stationary.

          \begin{sol}
              We have to check the three conditions from the definition of stationarity, given in the lecture.

              \textbf{Condition 1}, $\E[y_t] = \mu < \infty$:
              \begin{align*}
                  \E[y_t] = \E[\nu_t] + \psi \E[\nu_{t-1}]
                  = 0 + \phi \times 0
                  = 0
              \end{align*}
              Thus, the mean is constant and finite.

              \textbf{Condition 2}, $\Var(y_t) = \sigma^2 < \infty$:
              \begin{align*}
                  \Var(y_t)
                   & = \Var(\nu_t) + \psi^2 \Var(\nu_{t-1})
                  = 1 + \psi^2
              \end{align*}
              Thus, the variance is constant and finite.

              \textbf{Condition 3}, $\Cov(y_t, y_{t-k}) = \gamma_k, k>0$:
              \begin{align*}
                  \Cov(y_t, y_{t-k})
                   & = \Cov(\nu_t + \psi \nu_{t-1}, \nu_{t-k} + \psi \nu_{t-k-1})
                  \\
                   & = \Cov(\nu_t, \nu_{t-k})
                  + \Cov(\nu_t, \nu_{t-k-1})
                  + \Cov(\nu_{t-1}, \nu_{t-k})
                  + \Cov(\nu_{t-1}, \nu_{t-k-1})
                  \\
                   & \Downarrow
                  \\
                  \text{For $k=1$}: \quad \Cov(y_t, y_{t-1})
                   & = 0 + 0 + \Cov(\nu_{t-1}, \nu_{t-1}) + 0 = 1
                  \\
                  \text{For $k=2$}: \quad \Cov(y_t, y_{t-2})
                   & = 0 + 0 + 0 + 0 = 0
                  \\
                  \text{For $k=3$}: \quad \Cov(y_t, y_{t-3})
                   & = 0 + 0 + 0 + 0 = 0
                  \\
                   & \vdots
              \end{align*}
              Thus, the covariance between $y_t$ and $y_{t-k}$ is only a function of the time between two points, and not of time itself.
          \end{sol}

    \item Using R, simulate $2500$ observations from the \arp[1]{} process, $y_t = \phi y_{t-1} + \nu_t$ with $\nu_t \sim N(0, 1)$, for each $\phi = 0.99$, $\phi = 1.00$, and $\phi = 1.01$. Plot the simulated series and the corresponding auto-correlation function (ACF). Try and guess which cases are stationary and which cases are non-stationary. 

    \begin{sol}
        \lstinputlisting{../R-files/simulate_ar1.R}
    \end{sol}

    \item Consider an \armapq{} process, $\Phi(L) X_t = \Theta(L) \nu_t$:
        \begin{enumerate}[label=$\bullet$]
        \item \armapq{} processes are \emph{stable}, if all roots of 
        \[ \Phi(z) = 0 \]
        are larger than 1 in absolute values.

        \item \armapq{} processes are \emph{invertible}, if all roots of 
        \[ \Theta(z) = 0 \]
        are larger than 1 in absolute values.\footnote{For more details, see \cite[][Chapter 3, "Stationary ARMA Processes"]{Hamilton-1994}}
        \end{enumerate}

        Check whether the following \armapq[2, 2]{} is stable and invertible.
        \[
            X_t
            = 0.2 X_{t-1} + 0.48 X_{t-2}
            + \nu_t - 0.2 \nu_{t-1} - 0.08 \nu_{t-2}
        \]

        \begin{sol}
            \begin{align*}
                X_t
                & = 0.2 X_{t-1} + 0.48 X_{t-2}
                + \nu_t - 0.2 \nu_{t-1} - 0.08 \nu_{t-2}
                \\
                X_t - 0.2 X_{t-1} - 0.48 X_{t-2} 
                & = \nu_t - 0.2 \nu_{t-1} - 0.08 \nu_{t-2}
                \\
                (1 - 0.2 L - 0.48 L^2) X_t 
                & = (1 - 0.2 L - 0.08 L^2) \nu_t
            \end{align*}

            Using the following R script, we find that both $\Phi(z) = 0$ and $\Theta(z) = 0$ have roots outside the unit circle.
            \lstinputlisting{../R-files/polynomial_roots.R}
            Thus, the process is both stable and invertible.
        \end{sol}
\end{enumerate}